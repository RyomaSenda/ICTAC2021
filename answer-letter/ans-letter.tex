\documentclass{article}
\usepackage{adjustbox}
\usepackage{algorithm,algorithmic}
\usepackage{booktabs}
\usepackage{amsmath}
\usepackage{amssymb}
\usepackage{bmpsize} %%% senda
%\usepackage{amsthm}
\usepackage{booktabs}
\usepackage[shortlabels]{enumitem}
%\usepackage{forest}
\usepackage{latexsym}
%\usepackage{multirow}
%\usepackage{pgfplots}
%\usepackage{pgfplotstable}
%\usepackage{thmtools}
\usepackage{xspace}
\usepackage{here}

\newtheorem{example}{Example}
\newtheorem{definition}{Definition}
\newtheorem{theorem}{Theorem}
\newtheorem{lemma}{Lemma}
\newtheorem{corollary}{Corollary}
\newtheorem{property}{Property}
\newtheorem{proposition}{Proposition}
\newtheorem{assumption}{Assumption}
\newtheorem{condition}{Condition}
\newtheorem{claim}{Claim}

\newcommand{\stackcell}[2][c]{\begin{tabular}[#1]{@{}c@{}}#2\end{tabular}}
\newcommand{\Pros}[1]{\mathtt{#1}}
\newcommand{\Nat}{\mathbb{N}}
\newcommand{\Natz}{\mathbb{N}_0}
\newcommand{\done}{\Rightarrow}
\newcommand{\dast}{\stackrel{\ast}{\smash\Rightarrow\vphantom=}}
\newcommand{\dastr}{\stackrel{\ast}{\smash\Leftarrow\vphantom=}}
\newcommand{\dpls}{\stackrel{+}{\smash\Rightarrow\vphantom=}}
\newcommand{\vone}{\vdash}
\newcommand{\vast}{\stackrel{\ast}{\vdash}}
\newcommand{\vpls}{\stackrel{+}{\vdash}}
\newcommand{\inv}{\mathit{inv}}
\newcommand{\Comp}{\mathit{Comp}}
\newcommand{\Compl}{\mathit{Comp}_\mathit{L}}
\newcommand{\Compm}{\mathit{Comp}_\mathit{M}}
\newcommand{\Compw}{\mathit{Comp}}
\newcommand{\Compr}{\mathit{Comp}_\mathit{R}}
\newcommand{\Linked}{\hookrightarrow}
\newcommand{\Pre}{\mathit{pre}}
\newcommand{\Post}{\mathit{post}}
\newcommand{\PhiNew}{\Phi_{\mathit{new}}}
\newcommand{\deltaNew}{\delta_{\mathit{new}}}
\newcommand{\rNew}{r_{\mathit{new}}}
\renewcommand{\L}{{\tt x}}
\newcommand{\R}{{\tt x}'}
\newcommand{\TOP}{{\tt top}}
\newcommand{\INV}[1]{{#1}^{-1}}
\newcommand{\MCOMP}[2]{{#1}\mathbin{\circ_{\tt M}}{#2}}
\newcommand{\RCOMP}[2]{{#1}\mathbin{\circ_{\tt R}}{#2}}
\newcommand{\ANGLE}[1]{\langle{#1}\rangle}
\newcommand{\EXOR}{\not\Leftrightarrow}
\newcommand{\halfquad}{\hspace{.5em}}
\newcommand{\negqquad}{\hspace{-4em}}
\newcommand{\lefthang}[1]{\par\indent\llap{#1\ }~\ignorespaces}
\newcommand{\Acc}{\mathit{Acc}}
\newcommand{\Lat}{\mathit{lat}}
\newcommand{\ppsi}{^p_\psi}
\newcommand{\pps}{^p_A}
\newcommand{\TT}{\mathrm{tt}}
\newcommand{\X}{\mathcal{X}}
\newcommand{\U}{\mathbin{\mathcal{U}}} % \U is a binary operator
\newcommand{\Nzero}{\mathbb{N}_0}
\def\calP{{\cal P}}
\def\calA{{\cal A}}
\def\calT{{\cal T}}
\def\calTP{{\cal T}{\cal P}}
\def\calL{{\cal L}}
\def\calR{{\cal R}}
\def\calB{{\cal B}}
\def\calM{{\cal M}}
\def\calX{{\cal X}}
\def\calU{{\cal U}}
\def\calF{{\cal F}}
\def\calG{{\cal G}}
\def\calBP{{{\cal B}{\cal P}}}
\def\calBPR{{{\cal B}{\cal P}{\cal R}}}
\def\bolp{{\boldsymbol p}}
\def\bolq{{\boldsymbol q}}
\def\boltheta{{\boldsymbol \theta}}
\def\bolTheta{{\boldsymbol \Theta}}
\def\bolvartheta{{\boldsymbol \vartheta}}
\def\bolphi{{\boldsymbol \phi}}
\def\bolPhi{{\boldsymbol \Phi}}
\def\bolQ{{\boldsymbol Q}}
\def\bolB{{\boldsymbol B}}
\def\bols{{\boldsymbol s}}
\def\bolf{{\boldsymbol f}}
\def\States{\mathit{States}}
\def\tildeCalA{\widetilde{\cal A}}
\def\GOOD{\mathit{Good}}
\def\BAD{\mathit{Bad}}
\def\rep{\mathit{rep}}
\def\proj{\mathit{proj}}
\def\str{\mathit{str}}
\def\cor{\mathit{cor}}
\def\At{\mathit{At}}
\def\ID{\mathit{ID}}
\def\Check{\mathit{Check}}
\def\yield{\Rightarrow}
\def\tta{\mathtt{a}}
\def\ato{\stackrel{\tta}{\to}}
\def\eto{\stackrel{\epsilon}{\to}}
\def\top{\mathit{top}}
\def\pre{\mathit{pre}}
\def\post{\mathit{post}}
\newcommand{\Buchi}{B\"{u}chi\ }

% \makeatletter
% \newcommand{\xRightarrow}[2][]{%
% \ext@arrow 0055{\Rightarrowfill@}{#1}{#2}%
% }
% \def\Rightarrowfill@{\arrowfill@\Relbar\Relbar\Rightarrow}
% \newcommand{\xLeftarrow}[2][]{%
% \ext@arrow 0055{\Leftarrowfill@}{#1}{#2}%
% }
% \def\Leftarrowfill@{\arrowfill@\Leftarrow\Relbar\Relbar}
% \newcommand{\xLongleftrightarrow}[2][]{%
% \ext@arrow 0055{\llrafill@}{#1}{#2}%
% }
% \def\llrafill@{\arrowfill@\Leftarrow\Relbar\Rightarrow}
% \makeatother


\makeatletter
\newcommand{\xRightarrow}[2][]{%
\ext@arrow 0055{\Rightarrowfill@}{#1}{#2}%
}
\def\Rightarrowfill@{\arrowfill@\Relbar\Relbar\Rightarrow}
\newcommand{\xLeftarrow}[2][]{%
\ext@arrow 0055{\Leftarrowfill@}{#1}{#2}%
}
\def\Leftarrowfill@{\arrowfill@\Leftarrow\Relbar\Relbar}
\newcommand{\xLongleftrightarrow}[2][]{%
\ext@arrow 0055{\llrafill@}{#1}{#2}%
}
\def\llrafill@{\arrowfill@\Leftarrow\Relbar\Rightarrow}
\makeatother

\title{Answers to the reviewers' comments on \\``LTL Model Checking for Register Pushdown Systems''}
\date{}

\begin{document}
\pagenumbering{gobble}
\maketitle
The authors would like to thank the reviewers
for their thoughtful comments on the previous version.
All comments and suggestions have been taken into consideration
and the authors have revised the paper according to the comments as follows.
%\newline
%In the marked paper,
%the part revised according to the reviewers' comments are marked in three colors:
%yellow for reviewer A,
%blue for reviewer B, and green for both.

\newpage
\pagenumbering{arabic}

\section*{Reviewer A}

\paragraph{(A1)} Proposition 3.2. \newline
\textit{
It should be stated that this proposition and the property (1') , (2') in page 6 are an adaptation
of Proposition 3.1 of [6] for pushdown systems.
}
\subparagraph{Answer:}
Thank you for the comment.
We stated the above mentioned proposition and properties are extensions of [6] as below.
\medskip\noindent\\
(Before Proposition 3.2, p.5, left)
This proposition is a natural but non-trivial extension of
[6, Proposition 3.1] to BRPDS.
\medskip\noindent\\
(Before properties (1') and (2'), p.6, left)
Using $\Pre^*$, $\Pre^+$, $L_{\phi}^p$ and $L_{\phi,1}^p$,
Proposition~3.2 can be rephrased as follows,
which is a BRPDS version of (1') and (2') in
[6, Proposition~3.1]:

\paragraph{(A2)}
\textit{
It should be mentioned that LTL model checking for regular valuations in this paper takes the
approach of Technique 2 of [16].
}
\subparagraph{Answer:}
According to the comment,
we revised the first sentence of the paragraph before Proposition 6.2.
\medskip\noindent\\
(Previous version) The idea behind the proof is as follows.\\
(Revised version) The idea behind the proof is based on [16], but we need
to extend the idea of [16] to deal with RPDS as follows.
\medskip\noindent\\

\paragraph{(A3)}
\textit{
It should be clarified that it can handle only deterministic
regular valuations since register automata cannot be determinized. See the comment below.
}
\subparagraph{Answer:}
Thank you for the comment.
In Introduction of the previous version, we described that register automata cannot be determinized and the constraint ``backward-deterministic'' is essencial.
However, we did not clarify that regular valuations for RPDS are deterministic and cannot be determinized.
In the revised paper, the underlined sentence is added in Introduction (before Related Work, p.2, left).

\begin{quote}\em
	In the algorithm for regular valuations,
we assume backward-deterministic RAs are available for representing the regular pattern of the stack contents.
This constraint is essential because unlike NFAs, determinization is not possible for general RAs.
	\underline{Thus, regular valuations for}
      \underline{RPDS can only handle deterministic cases, not as [16].}
\end{quote}

\paragraph{(A4)} Section 3.2, Definition 3.3, Definition of $k$-RA \newline
\textit{
$k$-RA are used for the following two purposes.
\medskip\noindent\\
(1) to represent a regular set of configurations\\
(2) to represent a regular valuation
\medskip\noindent\\
The automata used in [16] for (2) are different from this definition in the sense that an
automaton in this paper reads stack from the top, but that of [16] reads stack from the
bottom. This should be clarified.
\medskip\noindent\\
Related to the difference above, `backward-deterministic' in this paper and [16] are different.
The restriction of `backward-deterministic' in this paper corresponds to deterministic in [16].
This should be clarified.
}
\subparagraph{Answer:}
Thank you for the comment.
To clarify the difference between regular valuations of this paper and
those of [16], we added the following sentences after the definition of regular valuations (Definition 4.2, p.6, right).

\begin{quote}\em
	The definitions of simple and regular valuations are extensions of those of [6] and [16], respectively.
	We define a regular valuation for RPDS
	based on backward-deterministic regular sets of $\ID_\calP$,
	which corresponds to the regular valuation for PDS in [16]
	based on deterministic finite automata
	that decide the acceptance of the reverse of stack contents.
\end{quote}


\paragraph{(A5)} Definition 4.3 \newline
\textit{
It will be more precise to say that $\ID_\calP\to\Sigma$ is `backward-deterministic' or `deterministic'
regular.
}
\subparagraph{Answer:}
As the reviewer pointed out,
the valuation $\ID_\calP\to\Sigma$ in Definition 4.3
is better to say `backward-deterministic' or `deterministic' regular
because the valuation is specified by a backward-deterministic regular sets.
We revised Definition 4.3 as pointed out.
However, we continue to use the name ``regular valuation'' as the previous version of the paper for simplicity.

\paragraph{(A6)} p.6. Property (2') \newline
\textit{
I think that it is better to rephrase it as follows because there are infinitely many $\theta_0$ and $d$.
\begin{align*}
\pre^+_{\calBP}(\cdots)\cap L^p_{\phi, 1}\neq \emptyset
\end{align*}
This is related to the following sentence in p.10, right, -7.
\medskip\noindent\\
which is the membership problem of $(p, \theta_0, d_0)$ for a $k$-RA $\calA$
\medskip\noindent\\
I think that it is not appropriate to consider the problem for infinitely many $d_0$.
}
\subparagraph{Answer:}
Just as the reviewer said,
we intend to decide ($2'$) by testing the non-emptiness
of $\Pre^+_{\calBP}(\ldots)\cap L_{\phi,1}^p$,
instead of searching
the infinite set $L_{\phi,1}^p$ for $(p,\theta_0,d_0)$.
We mentioned $(p,\theta_0,d_0)$ in ($2'$) because
in the next paragraph, we explained why
we do not need to test whether
$(p,\theta_0,d_0w')$ mentioned in ($2'$) is reachable from
$(q_0,\theta,w)$.

In the revised paper,
we added a sentence just after ($2'$) that says
($2'$) is equivalent to the non-emptiness of a language.
Moreover, we revised the paragraph next to ($2'$)
so that the main point of the paragraph becomes clearer.
The revised paragraph next to ($2'$) is as follows:
\begin{quote}\em
  Obviously, ($2'$) is equivalent to
  $\Pre_{\calBP}^+((G\times\Theta_k\times D^*)\cap
  \Pre_{\calBP}^*(L_{\phi}^p))
  \cap L_{\phi,1}^p \neq\emptyset$.
  %
  Note that although Proposition 3.2
  requires that
  $(p,\theta_0,d_0)$ mentioned in ($2'$) should satisfy
  $(q_0,\theta,w)\done^* (p,\theta_0,d_0w')$
  for some $w'\in D^*$,
  we do not need to directly test it,
  because by Lemma 3.1,
  if $(p,\theta_0,d_0) \done^+ (g,\theta'_0,u_0) \done^*
  (p,\theta_1,d_1v_1)$
  for some %$(p,\theta_0,d_0)\in L_{\phi,1}^p$,
  $(g,\theta'_0,u_0)\in G\times\Theta_k\times D^*$
  and $(p,\theta_1,d_1v_1)\in L_{\phi}^p$,
  then
  for \emph{every}
  $(p,\theta_2,d_2)\in L_{\phi,1}^p$,
  there exist $(g,\theta'_2,u_2)\in G\times\Theta_k\times D^*$
  and $(p,\theta_3,d_3v_3)\in L_{\phi}^p$
  such that
  $(p,\theta_2,d_2) \done^+ (g,\theta'_2,u_2) \done^*
  (p,\theta_3,d_3v_3)$.
\end{quote}


\paragraph{(A7)} Proof of Proposition 6.2. \newline
\textit{
The construction in [16] extends the stack alphabet as $\Gamma'=\Gamma\times\States$.
On the other hand, this paper represents one symbol in the extended stack alphabet $\Gamma'$ with a sequence of $D$. I think that this complicates the proof of this proposition. It should be clarified why you take this approach.
}
\subparagraph{Answer:}
Thank you for the comment.
If we extend stack alphabet to $D\times\States$, like $\Gamma'=\Gamma\times\States$ in [16],
we also need an extension of the definition of RPDS
because the definition of RPDS only allows its stack to
store data values in $D$ from its registers.
In order to avoid an extension of RPDS,
we take the way to simulate every symbol in $\States$
by using the encoding table $\theta_\bolQ$ in the proof of the proposition.

As the reviewer pointed out,
the previous version of this paper has no explanation why
we do not use the same approach as in [16].
Thus, we added the following sentence at the end of the paragraph before Proposition 6.2.

\begin{quote}\em
	More accurately, we introduce the encoding table $\theta_Q$ that keeps distinct data values $d_{p_1},\ldots,d_{p_{|Q|}}$ and
	we store the constant data value $d_{p_i}$ instead of the state $p_i$ of $\calA$
	to registers and stack of $\calP'$ because we cannot store a certain finite symbol $p_i$ without extending the definition of RPDS.
\end{quote}


\paragraph{(A8)}
\textit{
I think that it is better that the construction in the proof of Proposition 6.2 is presented in the
main text because LTL model checking for regular valuations is one of the main contributions
of this paper.
}
\subparagraph{Answer:}
Thank you for the comment.
According to the comment,
we move the full proof of Proposition 6.2 to the main text.


\paragraph{(A9)} p.3, left, l.5. ``$d_1 \ldots d_k \equiv$" \newline
\textit{
It is more common to use $\cdots$ instead of $\ldots$.
}
\subparagraph{Answer:}
Although the authors used \verb+\ldots+ there according to the style
presented in the \TeX{}book [Knuth86],
we changed $d_1\ldots d_k$ to $d_1\cdots d_k$
and every abbreviation ``$x_1\ldots x_n$'' of concatenation
to ``$x_1\cdots x_n$'' through this paper.
\medskip\noindent\\\noindent
[Knuth86] D.E. Knuth, ``The \TeX{}book,'' 
Chap.18, Sect.5: Ellipses (``three dots''), pp.171--172,
Addison-Wesley, 1986.

\paragraph{(A10)} p.3, right, l.-9. ``$(p, \theta, w)$ can transit to $(q, \theta', w')$" \newline
\textit{
I think that it is more common to say that $(q, \theta', w')$ is a successor of
$(p, \theta. w)$. Please check the usage.
}
\subparagraph{Answer:}
Thank you for the comment.
We changed the sentence to ``$(q, \theta', w')$ is a successor of $(p, \theta, w)$'' as the reviewer pointed out.


\paragraph{(A11)} p.5, Section 3.2, the definition of regular languages.
``We call a set $C$ ... regular if there exists a $k$-RA $\calA$ ..."\newline
\textit{
It is not clear if $k$ is fixed or ``for some $k$".
}
\subparagraph{Answer:}
Thank you for the comment. We added the underlined sentence as follows and clarified that
$k$ is arbitrary but both $k$ which define $C\subseteq I\times\Theta_k\times D^*$ and $k$-RA $\calA$ are the same.

\begin{quote}\em
We call a set $C\subseteq I\times\Theta_k\times D^*$ \emph{regular} if there exists a $k$-RA $\calA$ that satisfies $C=L(\calA)$.
\underline{(The number $k$ of registers of $\calA$ equals the}
\underline{number of registers the set $C$ of IDs assumes.)}
\end{quote}

\paragraph{(A12)}  p.6, left, l.-8, ``We extend the domain of $\Lambda$ ...."\newline
\textit{
It is not clear what it means.
}
\subparagraph{Answer:}
Thank you for the comment. We added the definition of the extension of $\Lambda$ after the sentence and clarified the meaning as below.
\begin{quote}\em
  We extend the domain of $\Lambda$ to $(\ID_\calP)^\infty$ pointwise;
  that is, $\Lambda(c_1c_2\cdots)=\Lambda(c_1)\Lambda(c_2)\cdots$ for
  $c_1,c_2,\ldots\in \ID_\calP$.
\end{quote}


\paragraph{(A13)} p.6, right, l.-5. ``secure information" \newline
\textit{
It is not clear what it means. Maybe you mean ``security sensitive information".
}
\subparagraph{Answer:}
Thank you for the comment. We changed the sentence as the reviewer pointed out.

\paragraph{(A14)} p.7, left, l.18. ``We construct $\calP_1$ a ..."\newline
\textit{
It is better if you explain what the second register is used for.
}
\subparagraph{Answer:}
According to the comment, we inserted an explanation of the second register as below.
\begin{quote}\em
  We construct $\calP_{1a}$ and $\calP_{1b}$ as $3$-RPDS whose 1st and 2nd registers are used
  for keeping the contents of $\mathit{eax}$ and $\mathit{ebx}$, respectively,
  and the data value in the 3rd register stands for $0$ in the original program.
\end{quote}

\paragraph{(A15)} p.10, right, l.14. ``the abovementioned algorithm"\newline
\textit{
The algorithm is mentioned in page 6.
}
\subparagraph{Answer:}
As commented, we corrected this phrase as ``the algorithm described in Section 3.5".

\paragraph{(A16)} p.12, left, l.1.\newline
\textit{
It is not clear what $(p, \theta, d_n) \vdash ((\bolp_n)_A, (\boltheta_n)_A, \varepsilon)$ means.
}
\subparagraph{Answer:}
Thank you for the comment.
The symbols $(\bolp_n)_A$ and $(\boltheta_n)_A$
are the $A$ component of the tuples $\bolp_n$ and $\boltheta_n$,
which are the first elements of the consistent sequences
$\bolp_n,\ldots,\bolp_1$ and $\boltheta_n,\ldots,\boltheta_1$ of $(p,\theta,d_n\cdots d_1)$.
Thus, $(p, \theta, d_n) \vdash ((\bolp_n)_A, (\boltheta_n)_A, \varepsilon)$
iff $A\in\Lambda(p,\theta,d_n\cdots d_1)$ holds
by the definition of consistent sequences.
In the revised paper, we added these explanations to the sentence as below.

\begin{quote}\em
	By the definition of $\Lambda$ and $\calA_A$ for $A\in\At$,
	$A\in\Lambda(p,\theta,d_n\cdots d_1)$ iff
	$(p,\theta,d_n\cdots d_1)\in L(\calA_A)$ iff
	$(p,\theta,d_n)\vdash_{\calA_A}
	((\bolp_n)_A,(\boltheta_n)_A,\varepsilon)$ holds
	where
	$\bolp_n$ and $\boltheta_n$ are the first elements of
	the consistent sequences $\bolp_n,\ldots,\bolp_1$ and $\boltheta_n,\ldots,\boltheta_1$ of $(p,\theta,d_n\cdots d_1)$ and
	$(\bolp_n)_A$ and $(\boltheta_n)_A$ are the components of $\bolp_n$ and $\boltheta_n$ indexed by $A$, respectively.
\end{quote}

\paragraph{(A17)} p.12, left, l.8.\newline
\textit{
simulate a transition of $\calP'$ $\to$ simulate a transition of $\calP$
}
\subparagraph{Answer:}
As the reviewer pointed out, we modified $\calP'$ to $\calP$.

\paragraph{(A18)} P. 12, left, l.-24. ``is a bijection"\newline
\textit{
It is not a bijection. Maybe, you mean bijection to the consistent
configurations of $\ID_{\calP'}$.
}
\subparagraph{Answer:}
As the reviewer mentioned, $\cor: \ID_\calP \to \ID_{\calP'}$ is not a bijection.
We corrected the sentence as below.
\begin{quote}\em
  For a fixed encoding table $\theta_\bolQ$, $\cor: \ID_\calP \to \ID_{\calP'}$ defined above satisfies $\cor(c)=\cor(c')$ iff $c=c'$
  for any $c,c'\in\ID_\calP$.
\end{quote}


\newpage
\section*{Reviewer B}

\paragraph{(B1)}
\textit{
 The author uses $D_k$ for representing the words over $D$ of length $k$.
Is it standard in the theory of register automata? In formal language theory,
it is standard to use $D^k$ for representing it.
}
\subparagraph{Answer:}
Thank you for the comment.
The representation $D_k$ is not standard.
According to the comment, we changed $D_k$ to $D^k$.


%\section*{Other corrections}

%Additionally, we corrected the following mistakes in the previous version.

%\paragraph{(C1)} p.6, Line~15 of Alg.~2:
%\begin{tabbing}
%  \qquad\=(Corrected) \=\kill
%  \>(Previous)\>%
%  \textbf{until} {no more transitions and final states can be added.} \\
%  \>(Corrected)\>%
%  \textbf{until} {no more transitions and accepting conditions can be added.}
%\end{tabbing}

%\paragraph{(C2)} p.8, the proof of Lemma~5.3,
%  the last sentence of the first paragraph of the induction step:
%\begin{tabbing}
%  \qquad\=(Corrected) \=\kill
%  \>(Previous)\>%
%  The case that $h = 0$ can be easily proved. \\
%  \>(Corrected)\>%
%  The case that $m = 0$ can be easily proved.
%\end{tabbing}


\end{document}
