%\documentclass[11pt]{article}
%\usepackage{xr}  % refer to labels in other documents

%% <local definitions here>

%\input{packages}
%\newcommand{\stackcell}[2][c]{\begin{tabular}[#1]{@{}c@{}}#2\end{tabular}}
\newcommand{\Pros}[1]{\mathtt{#1}}
\newcommand{\Nat}{\mathbb{N}}
\newcommand{\Natz}{\mathbb{N}_0}
\newcommand{\done}{\Rightarrow}
\newcommand{\dast}{\stackrel{\ast}{\smash\Rightarrow\vphantom=}}
\newcommand{\dastr}{\stackrel{\ast}{\smash\Leftarrow\vphantom=}}
\newcommand{\dpls}{\stackrel{+}{\smash\Rightarrow\vphantom=}}
\newcommand{\vone}{\vdash}
\newcommand{\vast}{\stackrel{\ast}{\vdash}}
\newcommand{\vpls}{\stackrel{+}{\vdash}}
\newcommand{\inv}{\mathit{inv}}
\newcommand{\Comp}{\mathit{Comp}}
\newcommand{\Compl}{\mathit{Comp}_\mathit{L}}
\newcommand{\Compm}{\mathit{Comp}_\mathit{M}}
\newcommand{\Compw}{\mathit{Comp}}
\newcommand{\Compr}{\mathit{Comp}_\mathit{R}}
\newcommand{\Linked}{\hookrightarrow}
\newcommand{\Pre}{\mathit{pre}}
\newcommand{\Post}{\mathit{post}}
\newcommand{\PhiNew}{\Phi_{\mathit{new}}}
\newcommand{\deltaNew}{\delta_{\mathit{new}}}
\newcommand{\rNew}{r_{\mathit{new}}}
\renewcommand{\L}{{\tt x}}
\newcommand{\R}{{\tt x}'}
\newcommand{\TOP}{{\tt top}}
\newcommand{\INV}[1]{{#1}^{-1}}
\newcommand{\MCOMP}[2]{{#1}\mathbin{\circ_{\tt M}}{#2}}
\newcommand{\RCOMP}[2]{{#1}\mathbin{\circ_{\tt R}}{#2}}
\newcommand{\ANGLE}[1]{\langle{#1}\rangle}
\newcommand{\EXOR}{\not\Leftrightarrow}
\newcommand{\halfquad}{\hspace{.5em}}
\newcommand{\negqquad}{\hspace{-4em}}
\newcommand{\lefthang}[1]{\par\indent\llap{#1\ }~\ignorespaces}
\newcommand{\Acc}{\mathit{Acc}}
\newcommand{\Lat}{\mathit{lat}}
\newcommand{\ppsi}{^p_\psi}
\newcommand{\pps}{^p_A}
\newcommand{\TT}{\mathrm{tt}}
\newcommand{\X}{\mathcal{X}}
\newcommand{\U}{\mathbin{\mathcal{U}}} % \U is a binary operator
\newcommand{\Nzero}{\mathbb{N}_0}
\def\calP{{\cal P}}
\def\calA{{\cal A}}
\def\calT{{\cal T}}
\def\calTP{{\cal T}{\cal P}}
\def\calL{{\cal L}}
\def\calR{{\cal R}}
\def\calB{{\cal B}}
\def\calM{{\cal M}}
\def\calX{{\cal X}}
\def\calU{{\cal U}}
\def\calF{{\cal F}}
\def\calG{{\cal G}}
\def\calBP{{{\cal B}{\cal P}}}
\def\calBPR{{{\cal B}{\cal P}{\cal R}}}
\def\bolp{{\boldsymbol p}}
\def\bolq{{\boldsymbol q}}
\def\boltheta{{\boldsymbol \theta}}
\def\bolTheta{{\boldsymbol \Theta}}
\def\bolvartheta{{\boldsymbol \vartheta}}
\def\bolphi{{\boldsymbol \phi}}
\def\bolPhi{{\boldsymbol \Phi}}
\def\bolQ{{\boldsymbol Q}}
\def\bolB{{\boldsymbol B}}
\def\bols{{\boldsymbol s}}
\def\bolf{{\boldsymbol f}}
\def\States{\mathit{States}}
\def\tildeCalA{\widetilde{\cal A}}
\def\GOOD{\mathit{Good}}
\def\BAD{\mathit{Bad}}
\def\rep{\mathit{rep}}
\def\proj{\mathit{proj}}
\def\str{\mathit{str}}
\def\cor{\mathit{cor}}
\def\At{\mathit{At}}
\def\ID{\mathit{ID}}
\def\Check{\mathit{Check}}
\def\yield{\Rightarrow}
\def\tta{\mathtt{a}}
\def\ato{\stackrel{\tta}{\to}}
\def\eto{\stackrel{\epsilon}{\to}}
\def\top{\mathit{top}}
\def\pre{\mathit{pre}}
\def\post{\mathit{post}}
\newcommand{\Buchi}{B\"{u}chi\ }

% \makeatletter
% \newcommand{\xRightarrow}[2][]{%
% \ext@arrow 0055{\Rightarrowfill@}{#1}{#2}%
% }
% \def\Rightarrowfill@{\arrowfill@\Relbar\Relbar\Rightarrow}
% \newcommand{\xLeftarrow}[2][]{%
% \ext@arrow 0055{\Leftarrowfill@}{#1}{#2}%
% }
% \def\Leftarrowfill@{\arrowfill@\Leftarrow\Relbar\Relbar}
% \newcommand{\xLongleftrightarrow}[2][]{%
% \ext@arrow 0055{\llrafill@}{#1}{#2}%
% }
% \def\llrafill@{\arrowfill@\Leftarrow\Relbar\Rightarrow}
% \makeatother


%\externaldocument{main}

\def\QED{\hfill$\Box$}

%\newtheorem{definition}{Definition}
%\newtheorem{theorem}{Theorem}
%\newtheorem{lemma}{Lemma}

\newcommand{\For}{\mathit{for}}

%% </local definitions here>

%\begin{document}

\paragraph{RA with $\epsilon$-transitions}
%\paragraph*{(Preliminaries)}
%We define $\Phi_k{\downarrow}_{\L,\R}$
%as the set of equivalence relations over
%$\{\L_1,\ldots,\L_k,\R_1\ldots,\R_k\}$.
%For $\phi\in\Phi_k{\downarrow}_{\L,\R}$ and
For $\phi\in\Phi_k$ and
$\theta,\theta'\in\Theta_k$,
we define $\theta,\theta'\models_{\L,\R}\phi$ iff
for all $i, j\in[k]$,
\begin{align*}
  \L_i\equiv_{\phi} \L_j &\Leftrightarrow \theta(i)=\theta(j), \\
  \L_i\equiv_{\phi} \R_j &\Leftrightarrow \theta(i)=\theta'(j), \\
  \R_i\equiv_{\phi} \R_j &\Leftrightarrow \theta'(i)=\theta'(j).
\end{align*}


%\subsection*{($\epsilon$-transitions)}
We allow an RA to have an \emph{$\epsilon$-transition}
$(p,\phi)\eto q$.
%where $\phi\in\Phi_k{\downarrow}_{\L,\R}$.
Let $\calA=(Q,I,\Acc,\delta)$ be an RA\@.
Then, $(p,\theta,w)\vdash_{\calA} (q,\theta',w)$
iff there exists
${(p,\phi)\eto q}\in\delta$
% and $d\in D$ and $u\in D^*$
such that
$\theta,\theta'\models_{\L,\R}\phi$. % and $w=du$.

%\begin{definition}
The {\em former image} of an equivalence relation
%$\phi\in\Phi_k{\downarrow}_{\L,\R}$ is
$\phi\in\Phi_k$ is
$\For(\phi)\in\Phi_k{\downarrow}_{\L}$
that satisfies $\L_i \equiv_{\phi} \L_j \Leftrightarrow \L_i \equiv_{\For(\phi)} \L_j$ for all $i,j\in[k]$.
%\end{definition}


\setcounter{algorithm}{2}
\begin{algorithm}[h]
\caption{Algorithm for eliminating $\epsilon$-transitions of RA}
\label{alg3}
\begin{algorithmic}[1]
  \STATE Input: $k$-RA $\calA=(Q, P, \Acc, \delta)$
  \STATE Let $k$-RA $\calA_0=(Q, P, \Acc_0, \delta_0)$ where
  $\Acc_0=\Acc$ and
  $\delta_0=\delta$.
  \STATE $h := 0$.
  \REPEAT
  \IF{${(p,\phi_1)\eto q_1}\in \delta_h$
    and $(q_1,\Lat(\phi_1))\in\Acc_h$}
  \STATE Let $\Acc_{h+1}=\Acc_h\cup\{(p,\For(\phi_1))\}$.
  \ENDIF
  \IF{${(p,\phi_1)\eto q_1}\in \delta_h$ and
     ${(q_1,\phi_2)\to q_2}\in \delta_h$}
  \STATE Let $\delta_{h+1}=\delta_h\cup\{(p,\phi_{12})\to q_2\mid 
    \phi_{12}\in\RCOMP{\phi_1}{\phi_2}\}$.
  \ENDIF
  \STATE $h := h + 1$.
  \UNTIL {no more transitions and accepting conditions can be added.}
  \STATE Remove all $\epsilon$-transitions from $\delta_h$.
  \STATE Output: $\tildeCalA := \calA_h$
\end{algorithmic}
\end{algorithm}

%\paragraph*{Algorithm for $\epsilon$-transition elimination}
%Repeat the following until no more transitions
%and accepting conditions can be added:
%\begin{itemize}
%\item
%  If ${(p,\phi_1)\eto q_1}\in \delta$
%    and $(q_1,\Lat(\phi_1))\in\Acc$, then
%    add $(p,\For(\phi_1))$ into $\Acc$.
%\item
%  If ${(p,\phi_1)\eto q_1}\in \delta$ and
%     ${(q_1,\phi_2)\to q_2}\in \delta$, then
%    add $(p,\phi_{12})\to q_2$ into $\delta$ for
%    every $\phi_{12}\in\RCOMP{\phi_1}{\phi_2}$.
%\end{itemize}
%After that, remove all $\epsilon$-transitions from
%$\delta$.

\paragraph*{Correctness of Algorithm~\ref{alg3}}
%Let $\calA_h$ be the RA obtained just after the $h$-th
%iteration of the above algorithm from an input RA~$\calA$.
%($\calA_0=\calA$.)

\begin{lemma}\label{lem:eps1}
$(p,\theta,\epsilon)$ is an accepting ID of $\calA_h$
for some $h\ge0$
iff
there exists an accepting ID $(q,\theta',\epsilon)$ of $\calA$
such that
$(p,\theta,\epsilon)\vdash_{\calA}^* (q,\theta',\epsilon)$.
\end{lemma}
%
(Proof)\quad
($\Rightarrow$) \ Induction on $h$.
The basis is trivial.
For the induction step,
assume that $(p,\theta,\epsilon)$ is an accepting ID of $\calA_h$
and not an accepting ID of $\calA_{h-1}$.
It implies that
there exist ${(p,\phi_1)\eto q_1}\in\delta$ and
$(q_1,\Lat(\phi_1))\in\Acc_{h-1}$, and
$\theta\models\For(\phi_1)$ holds.
This $\epsilon$-transition and $\theta\models\For(\phi_1)$ imply
$(p,\theta,\epsilon)\vdash_{\calA} (q_1,\theta_1,\epsilon)$
for some $\theta_1$ satisfying $\theta,\theta_1\models_{\L,\R}\phi_1$.
Since $\theta,\theta_1\models_{\L,\R}\phi_1$ implies
$\theta_1\models\Lat(\phi_1)$,
$(q_1,\theta_1,\epsilon)$ is an accepting ID of $\calA_{h-1}$.
Thus by the induction hypothesis,
there exists an accepting ID $(q,\theta',\epsilon)$ of $\calA$
such that
$(p,\theta,\epsilon)\vdash_{\calA} (q_1,\theta_1,\epsilon)
\vdash_{\calA}^* (q,\theta',\epsilon)$.

($\Leftarrow$) \
Induction on the length of the run
$(p,\theta,\epsilon)\vdash_{\calA}^* (q,\theta',\epsilon)$.
The basis is trivial.
For the induction step, assume that
$(p,\theta,\epsilon)\vdash_{\calA}(q_1,\theta_1,\epsilon)
\vdash_{\calA}^* (q,\theta',\epsilon)$.
By the induction hypothesis,
$(q_1,\theta_1,\epsilon)$ is an accepting ID of
$\calA_h$ for some~$h$.
On the other hand,
$(p,\theta,\epsilon)\vdash_{\calA}(q_1,\theta_1,\epsilon)$
implies that
${(p,\phi_1)\eto q_1}\in\delta$ for some $\phi_1$
satisfying $\theta,\theta_1\models_{\L,\R}\phi_1$.
It implies that $\theta_1\models\Lat(\phi_1)$,
and thus $(q_1,\Lat(\phi_1))\in\Acc_h$.
Therefore $(p,\For(\phi_1))\in\Acc_{h+1}$, and hence
$(p,\theta,\epsilon)$ is an accepting ID of $\calA_{h+1}$.
\qed

% % %

\begin{lemma}[A modified version of Lemma~3.3]
\label{lem: 4eps}
Let $\theta_1,\theta_3 \in \Theta_k$, $d\in D$ and $\phi_1,\phi_2 \in \Phi_k$.
\lefthang{(i)}
If there exist $\theta_2\in\Theta_k$ such that $\theta_1,\theta_2\models_{\L,\R} \phi_1$
and $\theta_2,\theta_3,d\models \phi_2$, then
$\phi_{12}\in \RCOMP{\phi_1}{\phi_2}$ holds for the unique $\phi_{12}\in \Phi_k$ satisfying
$\theta_1,\theta_3,d\models \phi_{12}$.
\lefthang{(ii)}
If there exists $\phi_{12}\in \RCOMP{\phi_1}{\phi_2}$ satisfying
$\theta_1,\theta_3,d\models \phi_{12}$, then
there exist $\theta_2\in\Theta_k$ such that
$\theta_1,\theta_2\models_{\L,\R} \phi_1$
and $\theta_2,\theta_3,d\models \phi_2$.
\end{lemma}
%
(This lemma can be proved in a similar way to
Lemma~3.3.)

% % %

\begin{lemma}\label{lem:eps2}
$(p,\theta,du)\vdash_{\calA_h} (q,\theta',u)$
for some $h\ge0$
iff
$(p,\theta,du)\vdash_{\calA}^* (q_1,\theta_1,du)
\vdash_{\calA} (q,\theta',u)$
for some $q_1$ and $\theta_1$.
\end{lemma}
%
(Proof)\quad
($\Rightarrow$) \ Induction on $h$.
The basis is trivial.
For the induction step,
let $(p,\phi)\to q$ be the transition rule
that enables
$(p,\theta,du)\vdash_{\calA_h} (q,\theta',u)$.
It implies that $\theta,\theta',d\models\phi$.
Assume that $(p,\phi)\to q$ is
not a transition rule of $\calA_{h-1}$.
It implies that
there exist ${(p,\phi_1)\eto q_1}\in\delta$ and
${(q_1,\phi_2)\to q}\in\delta_{h-1}$, and
$\phi\in\RCOMP{\phi_1}{\phi_2}$ holds.
By Lemma~\ref{lem: 4eps}\,(ii),
there exist $\theta_2\in\Theta_k$ such that
$\theta,\theta_2\models_{\L,\R} \phi_1$
and $\theta_2,\theta',d\models \phi_2$.
Hence,
$(p,\theta,du)\vdash_{\calA} (q_1,\theta_2,du)
\vdash_{\calA_{h-1}} (q,\theta',u)$.
Thus by the induction hypothesis,
$(p,\theta,du)\vdash_{\calA} (q_1,\theta_2,du)
\vdash_{\calA}^* (q,\theta',u)$.

($\Leftarrow$) \
Induction on the length of the run
$(p,\theta,du)\vdash_{\calA}^* (q_1,\theta_1,du)$.
The basis is trivial.
For the induction step, assume that
$(p,\theta,du)\vdash_{\calA}(q_2,\theta_2,du)
\vdash_{\calA}^* (q_1,\theta_1,du)
\vdash_{\calA} (q,\theta',u)$.
By the induction hypothesis,
$(q_2,\theta_2,du)\vdash_{\calA_h} (q,\theta',u)$
for some~$h$.
It implies that
there exists a transition rule
${(q_2,\phi_2)\to q}\in\delta_{h}$ satisfying
$\theta_2,\theta',d\models\phi_2$.
On the other hand,
$(p,\theta,du)\vdash_{\calA}(q_2,\theta_2,du)$
implies that
${(p,\phi_1)\eto q_2}\in\delta$ for some $\phi_1$
satisfying $\theta,\theta_2\models_{\L,\R}\phi_1$.
By Lemma~\ref{lem: 4eps}\,(i),
$\phi_{12}\in\RCOMP{\phi_1}{\phi_2}$ holds for the unique
$\phi_{12}$ that satisfies $\theta,\theta',d\models\phi_{12}$.
Therefore,
${(p,\phi_{12})\to q}\in\delta_{h+1}$ and
$(p,\theta,du)\vdash_{\calA_{h+1}} (q,\theta',u)$.
\qed

\medskip
By Lemmas \ref{lem:eps1} and \ref{lem:eps2},
we can conclude that $L(\calA)=L(\widetilde{\calA})$.



%\end{document}
