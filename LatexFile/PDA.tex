\section{Pushdown Transducers, Pushdown Automata and Pushdown Games}
\subsection{Pushdown Transducers}
\begin{definition}
A {pushdown transducer} (PDT)
over finite alphabets $\Sigma_\dblI$, $\Sigma_\dblO$ and $\Gamma$
is $\calT=(P, p_0, z_0, \Delta)$ where
$P$ is a finite set of states,
$p_0\in P$ is the initial state,
$z_0\in \Gamma$ is the initial stack symbol and
$\Delta: P\times \Sigma_\dblI \times \Gamma \to P\times \Sigma_\dblO \times \Com(\Gamma)$ is a finite set of deterministic transition rules having one of the following forms:
\begin{itemize}
\item $(p, a, z) \rightarrow (q, b, \pop)$ \quad (pop rule)
\item $(p, a, z) \rightarrow (q, b, \skip)$ \quad (skip rule)
\item $(p, a, z) \rightarrow (q, b, \push(z))$ \quad (push rule)
% \item $(p, a, z) \rightarrow (q, \varepsilon, Z)$ \quad (non-$\varepsilon$ rule)
% \item $(p, \varepsilon, z) \rightarrow (q, b, Z)$ \quad ($\varepsilon$ rule)
\end{itemize}
where $p, q\in P$, $a\in\Sigma_\dblI$, $b\in\Sigma_\dblO$ and $z\in\Gamma$.
\end{definition}
\noindent
For a state $p\in P$ and
a stack $w \in \Gamma^*$,
$(p, w)$ is called
a {\em configuration} or {\em instantaneous description (abbreviated as ID)} of PDT $\calT$. Let $\ID_\calT$ denote the set of all IDs of $\calT$.
For two IDs $(p, w), (q, w')\in \ID_\mathcal{T}$ and $(a,b)\in\Sigma_\dblI \times \Sigma_\dblO$,
we say that $(p, w)$ can transit to $(q, w')$ with an input $a$ and an output $b$, written as $(p, w) \done_{\calT}^{a, b} (q, w')$, if there exist a rule $r=(p, a, z) \rightarrow (q, b, \com) \in \Delta$ and a sequence $u\in \Gamma^*$ such that $w=zu$ and $w'=Z(\com,z)u$.
If $\calT$ is clear from the context,
we abbreviate
$\done_{\calT}^{a, b}$ as $\done^{a, b}$.
If a sequence of IDs $(q_0, w_0), \cdots, (q_n, w_n)\in \ID_\calT$
and $a_1, \cdots, a_n\in\Sigma_\dblI, b_1, \cdots, b_n\in \Sigma_\dblO$ satisfy $(q_{i-1}, w_{i-1})\done^{a_i, b_i}(q_i, w_i)$ for all $i\in[n]$, we write $(q_0, w_0)\done^{w^\dblI, w^\dblO}(q_n, w_n)$
where $w^\dblI = a_1 \cdots a_n$ and $w^\dblO = b_1 \cdots b_n$.

% If we emphasize the rule $r$ and the data value $d$,
% we write $\done_d^r$.
By definition, any ID $(p, \varepsilon)\in\ID_{\calT}$ has
no successor.
That is, there is no transition from an ID with empty stack.
A {\em run} of PDT $\calT$ is a pair $(\rho, w)$ of an infinite sequence of IDs $\rho\in (\ID_\calT)^\omega$ and $w = a_1 b_1 \cdots \in (\Sigma_\dblI\cdot\Sigma_\dblO)^\omega$ that satisfy $\rho(i-1)\done_\calT^{a_i, b_i}\rho(i)$ for $i\geq 1$.
Let $\RUN_\calT$ denote the set of all runs of $\calT$.
We define $L(\calT) = \{w\in (\Sigma_\dblI\cdot\Sigma_\dblO)^\omega\mid
\text{there exists a run}\allowbreak \text{ $(\rho, w)$ for some $\rho\in(\ID_\calT)^\omega$ where $\rho(0)=(p_0, z_0)$}\}$.
Let \PDT\ be the class of PDT.
% We call a PDT $\calT=(P, p_0, z_0, \Delta)$ deterministic when $|\Delta(p,a,z)|\leq 1$ and complete when $|\Delta(p,a,z)|\geq 1$ for all $q\in Q, a\in\Sigma_\dblI$ and $z\in\Gamma$.
% Let $\PDT$ be the class of deterministic complete PDT.

% \begin{definition}
% A {pushdown automata} (PDA) over finite alphabets $\Sigma_\dblI$, $\Sigma_\dblO$ and $\Gamma$ is
% $\calA=(\calT, c)$ where
% $\calT=(P, p_0, z_0, \Delta)$ is a PDT and
% $c: Q \to [n]$ is a coloring function where $n\in \Nat$ is the number of priorities.
% \end{definition}

\subsection{Pushdown Automata}

\begin{definition}
A {pushdown automata} (PDA) over finite alphabets $\Sigma_\dblI$, $\Sigma_\dblO$ and $\Gamma$ is $\calA = (Q, Q_\dblI, Q_\dblO, q_0, z_0, c, \delta)$ where
$Q$, $Q_\dblI, Q_\dblO$ are finite sets of states that satisfy $Q=Q_\dblI\cup Q_\dblO$ and $Q_\dblI \cap Q_\dblO = \emptyset$,
$q_0\in Q_\dblI$ is the initial state,
$z_0\in \Gamma$ is the initial stack symbol,
$c: Q \to [n]$ is a coloring function where $n\in \Nat$ is the number of priorities and
$\delta: Q \times \Sigma \times \Gamma \to \scrP(Q \times \Com(\Gamma))$ is a finite set of transition rules, having one of the following forms:
\begin{itemize}
\item $(q_\dblI, a_\dblI, z) \rightarrow (q_\dblO, \com)$
\item $(q_\dblO, a_\dblO, z) \rightarrow (q_\dblI, \com)$
\end{itemize}
where $q_\dblI\in Q_\dblI, q_\dblO\in Q_\dblO, a_\dblI\in \Sigma_\dblI, a_\dblO\in \Sigma_\dblO, z\in \Gamma$ and $\com\in \Com(\Gamma)$.
\end{definition}

% In the rest of the paper,
% we denote a PDA $\calA = (\calT, c)$ as $\calA = (Q, q_0, z_0, \delta, c)$
% for $\calT = (Q, q_0, z_0, \Delta)$ and
% $\delta: P\times (\Sigma_\dblI \times \Sigma_\dblO) \times \Gamma \to \scrP (P \times \Com(\Gamma))$ satisfies
% $(p,a,z)\to (q,b,Z)\in\Delta$ iff $(p,(a,b),z)\to (q,Z)\in\Delta$
% for $p,q\in Q, a\in\Sigma_\dblI, b\in\Sigma_\dblO, z\in\Gamma$ and $Z\in \Com(\Gamma)$.

We define an ID, transition relation and
a run of $\calA$ in a similar way to those of $\calT$.
More concretely, we assume $\ID_\calA=\ID_\calT$,
$(q,u)\vdash^a_\calA(q',u')$ iff there exist a rule $(p, a, z) \rightarrow (q, \com) \in \delta$ and a sequence $u\in \Gamma^*$ such that $w=zu$ and $w'=Z(\com,z)u$.

A {\em run} of PDA $\calT$ is a pair $(\rho, w)$ of an infinite sequence of IDs $\rho\in (\ID_\calA)^\omega$ and $w = a_1 a_2 \cdots \in (\Sigma_\dblI\cdot\Sigma_\dblO)^\omega$ that satisfy $\rho(i-1)\done_\calT^{a_i}\rho(i)$ for $i\geq 1$.
Let $\RUN_\calA$ denote the set of all runs of $\calA$.
For $\rho \in (\ID_\calA)^\omega$,
let $\state(\rho)\subseteq Q$ be the set of states that occur infinitely in $\rho$.

We define $L_N(\calA)=\{w\in (\Sigma_\dblI \cdot \Sigma_\dblO)^\omega \mid \text{there exists a run }\allowbreak (\rho, w)\in\RUN_\calA\text{ where }\rho(0)=(p_0, z_0)\text{ and }\min_{q\in Q}\{c(q)\mid q\in\state(\rho)\}\text{ is even}\}$ and
$L_U(\calA)=\{w\in (\Sigma_\dblI \cdot \Sigma_\dblO)^\omega \mid \text{every run }\allowbreak (\rho, w)\in\RUN_\calA\text{ where }\rho(0)=(p_0, z_0)\text{ satisfies that }\min_{q\in Q}\{c(q)\mid q\in\state(\rho)\}\text{ is even}\}$.

We call a PDA $\calA=(P, p_0, z_0, \delta, c)$ deterministic if $|\delta(p,a,z)|\leq 1$ for all $p\in Q, a\in\Sigma$ and $z\in\Gamma$,
nondeterministic if the language of $\calA$ is defined as $L_N(\calA)$ and universal if the language of $\calA$ is defined as $L_U(\calA)$.
For those cases, we denote $\calA$ as DPDA, NPDA and UPDA, respectively.
Let $\DPDA, \NPDA$ and $\UPDA$ be the class of DPDA, NPDA and UPDA, respectively.

\subsection{Pushdown Games}

\begin{definition}
A {Pushdown Games} (PDG) of PDA $\calA = (Q, Q_\dblI, Q_\dblO, q_0, z_0, \delta, c)$ is $\calG_\calA = (V, V_\dblI, V_\dblO, E, C)$ where
$V = Q\times\Gamma^*$ is the set of vertices with $V_\dblI = Q_\dblI\times\Gamma^*, V_\dblO = Q_\dblO\times\Gamma^*$, $E\subseteq V\times V$ is the set of edges defined as $E = \{(v,v') \mid v \vdash^a v' \text{for some $a\in \Sigma$}\}$
and $C: V \to [n]$ is the coloring function such that
$C((q,u)) = c(q)$ for all $(q,u)\in V$.
\end{definition}

The game starts with some $(q_0,z_0)\in V_\dblI$.
When the current vertex is $v\in V_\dblI$,
Player I chooses a successor $v'\in V_\dblO$ of $v$ as the next vertice.
When the current vertex is $v\in V_\dblO$,
Player II chooses a successor $v'\in V_\dblI$ of $v$.
A finite or infinite sequence $\rho\in V^\infty$ is valid if
$\rho(0)=(q_0,z_0)$ and satisfy
$(\rho(i-1), \rho(i))\in E$ for every $i\geq 1$.
A play of $\calG_\calA$ is an infinite and valid sequence $\rho\in V^\omega$.
A play $\rho$ is winning for Player I iff $\state(\rho)$ is even.

By the definition of $\calG_\calA$,
every choice of a successor by players can be also expressed as
a choice of a pair $(q,\com)\in Q\times \Com(\Gamma)$. Furthermore, a choice of a successor can be expressed as a choice of $a\in\Sigma$ if $\calA$ is deterministic.
Thus, every valid sequence $\rho \in V^\infty$
corresponds one-to-one with a sequence $\tau\in (Q\times \Com(\Gamma))^\infty$.
In detail, for every $i\geq 0$, $\rho(i) = (q, zu), \tau(i) = (q', \com)$ and $\rho(i+1) = (q', Z(\com,z)u)$ hold for some $q, q'\in Q, z\in\Gamma, u\in\Gamma^*$ and $\com\in \Com(\Gamma)$.
We call $\tau$ valid if the corresponding $\rho$ is valid.

\begin{theorem}{[Walukiewucz, 2001]}
\label{the: wal}
If player I has a winning strategy of $\calG_\calA$,
we can construct a PDT $\calT$ over $Q_\dblI\times\Com(\Gamma), Q_\dblO\times\Com(\Gamma)$ and an stack alphabet $\Gamma'$ that gives a winning strategy of $\calG_\calA$.
That is, for every $\tau\in L(\calT)$, the corresponding play $\rho\in V^\infty$ is winning for Player I.
\end{theorem}

When $\calA$ is deterministic,
there is also a one-to-one correspondence between
a valid sequence $\rho\in V^\infty$ and a sequence of input and output alphabets $u\in\Sigma^\infty$.
In detail, for every $i\geq 0$, $\rho(i) = (q, zu), \rho(i+1) = (q', Z(\com,z)u)$ and $(q, u(i), z) \to (q', \com)\in\delta$ hold for some $q, q'\in Q, z\in\Gamma, u\in\Gamma^*$ and $\com\in \Com(\Gamma)$.

By the correspondence, the following lemma holds.
\begin{lemma}
\label{lem: 1}
A play $\rho$ is winning for Player I iff
the corresponding sequence $w\in \Sigma^\omega$ of $\rho$
satisfies $w\in L(\calA)$.
\end{lemma}

In a similar way to Theorem \ref{the: wal},
we can obtain the following lemma.
\begin{lemma}
\label{lem: 2}
If $\calA$ is deterministic and player I has a winning strategy of $\calG_\calA$,
we can construct a PDT $\calT$ over $\Sigma_\dblI, \Sigma_\dblO$ and $\Gamma'$ that gives a winning strategy of $\calG_\calA$.
That is, for every $w\in L(\calT)$, the corresponding play $\rho\in V^\infty$ is winning for Player I.
\end{lemma}
