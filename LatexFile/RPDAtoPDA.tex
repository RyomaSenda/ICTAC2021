\subsection{PDA simulating RPDA}

Let $\sigma:((\Sigma\cup\{\tau\})\times D)\to(\Sigma\cup\{\tau\})$
be the function
defined as $\sigma((a,d))=a$
for any $a\in\Sigma\cup\{\tau\}$ and $d\in D$.
In this subsection, we show that we can construct an NPDA $\calA'$
from a given $k$-NRPDA $\calA$ over $\SigmaI$, $\SigmaO$, $D$
such that $\sigma(L(\calA))=L(\calA')$.

Let $\Phi_k$ be the set of
\emph{equivalence relations}
over the set of $2k+1$ symbols
$X_k=\{
 \L_1,\L_2,\ldots,\L_k,
 \R_1,\R_2,\ldots,\R_k,
 \Ltop \}$.
 We write $a\equiv_{\phi}b$ and $a\not\equiv_{\phi}b$ to mean
 $(a,b)\in\phi$ and $(a,b)\notin\phi$, respectively,
for $a,b\in X_k$ and $\phi\in\Phi_k$.
%
Intuitively,
each $\phi\in\Phi_k$ represents the equality and inequality
among the data values in the registers and the stack top,
as well as the transition of the values in the registers
between two assignments.
Two assignments $\theta,\theta'$ and a value $d$ at the stack top
satisfy $\phi$,
denoted as $\theta,d,\theta'\models\phi$,
if and only if for $i,j\in[k]$,
\begin{align*}
  \L_i\equiv_{\phi}\L_j &\Leftrightarrow \theta(i)=\theta(j),
  & \L_i\equiv_{\phi}\Ltop &\Leftrightarrow \theta(i)=d, \\
  \L_i\equiv_{\phi}\R_j &\Leftrightarrow \theta(i)=\theta'(j),
  & \R_j\equiv_{\phi}\Ltop &\Leftrightarrow \theta'(j)=d , \\
  \R_i\equiv_{\phi}\R_j &\Leftrightarrow \theta'(i)=\theta'(j).
\end{align*}
%
Let $\phizero\in\Phi_k$ be the equivalence relation
satisfying $a\equiv_{\phizero}b$ for any $a,b\in X_k$.

For $\tst\subseteq [k]\cup\{\top\}$ and $\asgn\subseteq [k]$,
define a subset $\Phi_k^{\tst,\asgn}$ of $\Phi_k$ as:
\[
  \Phi_k^{\tst,\asgn} = \{ \phi\in\Phi_k \mid
    \begin{aligned}[t]
      & (\forall i\in\rlap{$\tst :{}$}\hphantom{\asgn:{}}
      \forall j\in [k]\cup\{\top\} :
      j\in\tst \Leftrightarrow \L_i \equiv_{\phi} \L_j), \\
      & (\forall i\in\asgn : \forall j\in [k]\cup\{\top\} :
      j\in\tst \Leftrightarrow \L_j \equiv_{\phi} \R_i), \\
      & (\forall i,j\in\asgn :
      \R_i \equiv_{\phi} \R_j), \\
      & (\forall i\in [k]\setminus\asgn :
      \L_i \equiv_{\phi} \R_i)\}.
    \end{aligned}
\]
For $j\in[k]$, define
$
  \Phi_k^{=,j} = \{ \phi\in\Phi_k \mid
    \Ltop \equiv_{\phi} \L_j, \
    \forall i\in[k] : \L_i \equiv_{\phi} \R_i \}
$.
By definition,
$\theta,e,\theta'\models\phi$ for
%$\theta,\theta'\in\Theta_k$, $e\in D$, and
$\phi\in\Phi_k^{\tst,\asgn}$
iff
$\theta,d,e\models\tst$ and
$\theta'=\theta[\asgn\gets d]$ for some $d\in D$.
Similarly,
$\theta,e,\theta'\models\phi$ for
$\phi\in\Phi_k^{=,j}$
iff
$\theta'=\theta$ and $\theta(j)=e$.

Let $\COMPBL$ and $\COMPBLT$ be binary relations over $\Phi_k$ defined as:
\begin{align*}
  \phi_1\COMPBL\phi_2 \ &:\Leftrightarrow \
      \R_i \equiv_{\phi_1} \R_j \Leftrightarrow \L_i \equiv_{\phi_2} \L_j
      \ \text{for } i,j\in[k].
  \\
%
  \phi_1\COMPBLT\phi_2 \ &:\Leftrightarrow \
      \phi_1\COMPBL\phi_2 \ \text{and} \
      \R_i \equiv_{\phi_1} \Ltop \Leftrightarrow \L_i \equiv_{\phi_2} \Ltop
      \ \text{for } i\in[k].
\end{align*}
%We say $\phi_1$ and $\phi_2$ are \emph{composable} if $\phi_1\COMPBLT\phi_2$.
Below we will define the \emph{composition} of two equivalence relations,
and $\phi_1\COMPBL\phi_2$ means that
$\phi_1$ and $\phi_2$ are composable.
%
For $\phi\in\Phi_k$ and $\Phi'\subseteq\Phi_k$,
let $\phi\COMPBL\Phi' = \{\phi'\in\Phi' \mid \phi\COMPBL\phi'\}$
and $\phi\COMPBLT\Phi' = \{\phi'\in\Phi' \mid \phi\COMPBLT\phi'\}$.
By definition,
$\phi\COMPBLT\Phi_k^{\tst,\asgn}$ consists of at most one equivalence relation
for any $\phi\in\Phi_k$, $\tst\subseteq [k]\cup\{\top\}$,
and $\asgn\subseteq[k]$.
Similarly,
$\phi\COMPBL\Phi_k^{=,j}$ consists of exactly one equivalence relation
for any $\phi\in\Phi_k$ and $j\in [k]$.


For $\phi_1,\phi_2\in\Phi_k$ with $\phi_1\COMPBL\phi_2$,
the \emph{composition} $\phi_1\COMP\phi_2$ of them
is the equivalence relation in $\Phi_k$ that satisfies the followings:
\begin{align*}
  \L_i \equiv_{\phi_1} \L_j &\Leftrightarrow
  \L_i \equiv_{\phi_1\COMP\phi_2} \L_j &&\text{for } i,j\in [k]\cup\{\top\},\\
  \R_i \equiv_{\phi_2} \R_j &\Leftrightarrow
  \R_i \equiv_{\phi_1\COMP\phi_2} \R_j &&\text{for } i,j\in [k],\\
  (\exists l\in [k] : \L_i \equiv_{\phi_1} \R_l \mathrel{\land}
  \L_l \equiv_{\phi_2} \R_j) &\Leftrightarrow
  \L_i \equiv_{\phi_1\COMP\phi_2} \R_j
  &&\text{for } i\in[k]\cup\{\top\},\ j\in [k].
\end{align*}
%
By definition,
if $\theta_1,d_1,\theta_2\models\phi_1$ and
$\theta_2,d_2,\theta_3\models\phi_2$,
then
$\theta_1\,d_1,\theta_3\models\phi_1\COMP\phi_2$.
By definition, $\COMP$ is associative.

%%%%%%%%%%
\smallskip

Let $\calA=(Q,\QI,\QO,q_0,\delta,c)$ be a $k$-NRPDA
over $\SigmaI$, $\SigmaO$, and $D$.
We construct a PDA
$\calA'=(Q',\QI',\QO',q'_0,\phizero,\delta',c')$
over $\SigmaI$, $\SigmaO$, and $\Phi_k$,
where $Q'=Q\times\Phi_k$, $\QI'=\QI\times\Phi_k$, $\QO'=\QO\times\Phi_k$,
$q'_0=(q_0,\phizero)$,
$c'((q,\phi))=c(q)$ for any $q\in Q$ and $\phi\in\Phi_k$,
and for any $(q,\phi_2)\in Q'$, $a\in\Sigma\cup\{\tau\}$,
and $\phi_1\in\Phi_k$,
$\delta'((q,\phi_2),a,\phi_1)$ is the smallest set
satisfying the following inference rules:
%We write $(q,a,\tst)\to(q',\asgn,\COM)\in\delta$ to mean
%$(q',\asgn,\COM)\in\delta(q,a,\tst)$.
%
\begin{align}
&
\begin{array}{l}
  \delta(q,a,\tst) \ni (q',\asgn,\skip), \
  \phi_1\COMPBL\phi_2, \
  \phi_3\in \phi_2\COMPBLT\Phi_k^{\tst,\asgn}
  \\ \hline
  \delta'((q,\phi_2),a,\phi_1) \ni ((q',\phi_2\COMP\phi_3),\skip)
\end{array}
%
\\[\medskipamount]
&
\begin{array}{l}
  \delta(q,a,\tst) \ni (q',\asgn,\pop), \
  \phi_1\COMPBL\phi_2, \
  \phi_3\in \phi_2\COMPBLT\Phi_k^{\tst,\asgn}
  \\ \hline
  \delta'((q,\phi_2),a,\phi_1) \ni ((q',\phi_1\COMP\phi_2\COMP\phi_3),\pop)
\end{array}
%
\\[\medskipamount]
&
\begin{array}{l}
  \delta(q,a,\tst) \ni (q',\asgn,\push(j)), \
  \phi_1\COMPBL\phi_2, \
  \phi_3\in \phi_2\COMPBLT\Phi_k^{\tst,\asgn}, \
  \phi_4\in \phi_3\COMPBL\Phi_k^{=,j}
  \\ \hline
  \delta'((q,\phi_2),a,\phi_1) \ni ((q',\phi_4),\push(\phi_2\COMP\phi_3))
\end{array}
%
\end{align}

Let $\calS_{\calA}=(\ID_{\!\calA},(q_0,\theta_{\bot},\bot),
\Sigma\times D,\{\tau\}\times D,{\vdash_{\!\calA}},c_{\calA})$
and $\calS_{\calA'}=(\ID_{\!\calA'},(q'_0,\phizero),\allowbreak
\Sigma,\{\tau\},{\vdash_{\!\calA'}},c_{\calA'})$ be the TSs
that represent the semantics of $\calA$ and $\calA'$, respectively.
Below we show that $\calS_{\calA}$ is $\sigma$-bisimilar to
$\calS_{\calA'}$.
We define an intermediate TS
$\calS_{\calA}^{\AUG}=
(\ID_{\!\calA}^{\AUG},(q_0,\theta_{\bot},(\bot,\theta_{\bot})),
\Sigma\times D,\{\tau\}\times D,{\vdash_{\!\calA^{\AUG}}},c'_{\calA})$
where
$\ID_{\!\calA}^{\AUG}=Q\times\Theta_k\times(D\times\Theta_k)^*$,
$c'_{\calA}((q,\theta,v))=c_{\calA}((q,\theta,\fst(v)))$ for
any $(q,\theta,v)\in\ID_{\!\calA}^{\AUG}$,
and $\vdash_{\!\calA^{\AUG}}$ is defined as follows:
$(q,\theta,v)\vdash_{\!\calA^{\AUG}}^{(a,d)}(q',\theta',v')$
if and only if
$\delta(q,a,\tst)\ni (q',\asgn,\com)$,
$d,\fst(v(0)),\theta\models\tst$,
$\theta'=\theta[\asgn\gets d]$, and
$v'= v(1{:})$, $v$, or $(\theta'(j'),\theta')v$
if $\com=\pop$, $\skip$, or $\push(j')$, respectively.
Obviously,
$\calS_{\calA}$ and $\calS_{\calA'}$ are bisimilar;
the smallest relation $R\subseteq \ID_{\!\calA}\times\ID_{\!\calA}^{\AUG}$
satisfying
$((q,\theta,\fst(v)),\allowbreak (q,\theta,v))\in R$
for every $(q,\theta,v)\in\ID_{\!\calA}^{\AUG}$
is a bisimulation relation
between $\calS_{\calA}$ and $\calS_{\calA}^{\AUG}$.
We can show the following lemma.
%
\begin{lemma}\label{lemma:bisim-rpda-pda}
$\calS_{\calA}^{\AUG}$ is $\sigma$-bisimilar to
$\calS_{\calA'}$.
\end{lemma}
\begin{proof}
Let $R\subseteq \ID_{\!\calA}^{\AUG}\times\ID_{\!\calA'}$
be the smallest relation
that satisfies
for every $q\in Q$, $\theta_0,\ldots,\theta_n\in\Theta_k$,
$d_0,\ldots,d_{n-1}\in D$, and $\phi_0,\ldots,\phi_n\in\Phi_k$,
$((q,\theta_n,(d_{n-1},\theta_{n-1})\allowbreak\ldots\allowbreak
(d_1,\theta_1)(d_0,\theta_0)),
((q,\phi_n),\phi_{n-1}\ldots\phi_1\phi_0))\in R$
if
$\forall i\in[n]: \theta_{i-1},d_{i-1},\theta_i\models\phi_i$ and
$\theta_{\bot},{\bot},\theta_0\models\phi_0$.
%
We can show that $R$ is a $\sigma$-bisimulation relation from
$\calS_{\calA}^{\AUG}$ to $\calS_{\calA'}$.

Assume that
$((q,\theta_n,v),\allowbreak ((q,\phi_n),u))\in R$
for $v=(d_{n-1},\theta_{n-1})\ldots(d_0,\theta_0)$ and
$u=\phi_{n-1}\ldots\phi_0$.
Because
$\forall i\in[n]: \theta_{i-1},d_{i-1},\theta_i\models\phi_i$ and
$\theta_{\bot},{\bot},\theta_0\models\phi_0$,
$\forall i\in[n]: \phi_{i-1}\COMPBL\phi_i$.
%
If
$(q,\theta_n,v)\vdash_{\calA^{\AUG}}^{(a,d)}(q',\theta',v')$,
then
there exist $\tst$, $\asgn$, $\com$, and $d$ such that
$\delta(q,a,\tst)\ni (q',\asgn,\com)$,
$\theta_n,d,d_{n-1}\models\tst$,
$\theta'=\theta[\asgn\gets d]$, and
$v'=v(1{:})$, $v$, or
$(\theta'(j'),\theta')v$
if $\com=\pop$, $\skip$, or $\push(j')$, respectively.
There exists $\phi'\in\phi_n\COMPBLT\Phi_k^{\tst,\asgn}$ because
$\theta_{n-1},d_{n-1},\theta_n\models\phi_n$ and
$\theta_n,d,d_{n-1}\models\tst$.
Thus, by definition,
$\delta'((q,\phi_n),a,\phi_{n-1})\allowbreak\ni ((q',\phi''),\com')$
where $\com'=\pop$ and $\phi''=\phi_{n-1}\COMP\phi_n\COMP\phi'$
if $\com=\pop$,
$\com'=\skip$ and $\phi''=\phi_n\COMP\phi'$ if $\com=\skip$, and
$\com'=\push(\phi_n\COMP\phi')$ and $\phi''\in\phi'\COMPBL\Phi_k^{=,j'}$
if $\com=\push(j')$.
Therefore,
$((q,\phi_n),u)\vdash_{\calA'}^a ((q',\phi''),u')$
where $u'=u(1{:})$, $u$, or
$(\phi_n\COMP\phi')u$ for $u=\phi_{n-1}\ldots\phi_0$
if $\com'=\pop$, $\skip$, or $\push(\phi_n\COMP\phi')$, respectively.
Because $\theta_{n-1},d_{n-1},\theta_n\models\phi_n$ and
$\theta_n,d_{n-1},\theta'\models\phi'$,
$\theta_{n-1},d_{n-1},\theta'\models\phi_n\COMP\phi'$.
In the case of $\com=\pop$,
$\theta_{n-2},d_{n-2},\theta'\models\phi_{n-1}\COMP\phi_n\COMP\phi'$
because $\theta_{n-2},d_{n-2},\theta_{n-1}\models\phi_{n-1}$ and
$\theta_{n-1},d_{n-1},\theta'\models\phi_n\COMP\phi'$.
In the case of $\com=\push(j')$,
$\theta',\theta'(j'),\theta'\models\phi''$
because $\phi''\in\Phi_k^{=,j'}$.
Hence, $((q',\theta',v'),((q',\phi''),\allowbreak u'))\in R$ in any case.

On the other hand,
if $((q,\phi_n),u)\vdash_{\calA'}^a ((q',\phi''),u')$, then
$\delta'((q,\phi_n),a,\allowbreak
\phi_{n-1})\ni ((q',\phi''),\com')$
where
$\com'=\pop$, $\skip$, or $\push(\phi''')$ if
$u'=u(1{:})$, $u$, or $\phi'''u$, respectively.
By definition,
there exist $\tst$, $\asgn$, $\com$, and
$\phi'\in\phi_n\COMP\Phi_k^{\tst,\asgn}$ such that
$\delta(q,a,\tst)\ni(q',\asgn,\com)$, and
$\com=\pop$ and $\phi''=\phi_{n-1}\COMP\phi_n\COMP\phi'$
if $\com'=\pop$,
$\com=\skip$ and $\phi''=\phi_n\COMP\phi'$ if $\com'=\skip$,
or
$\com=\push(j')$, $\phi''\in\phi'\COMPBL\Phi_j^{=,j'}$
and $\phi'''=\phi_n\COMP\phi'$ if $\com'=\push(\phi''')$.
Because $\theta_{n-1},d_{n-1},\theta_n\models\phi_n$ and
$\phi'\in\phi_n\COMPBLT\Phi_k^{\tst,\asgn}$,
there exists $d\in D$ satisfying
$\theta_n,d,d_{n-1}\models\tst$ and
$\theta_n,d_{n-1},\theta'\models\phi'$
for $\theta'=\theta_n[\asgn\gets d]$.
Therefore,
$(q,\theta_n,v)\vdash_{\calA^{\AUG}}^{(a,d)} (q',\theta',v')$
where
$v'=v(1{:})$, $v$, or
$(\theta'(j'),\theta')v$
if $\com=\pop$, $\skip$, or $\push(j')$, respectively.
We can show that $((q',\theta',v'),((q',\phi''),u'))\in R$
in the same way as the last paragraph.
\end{proof}

By Lemmas~\ref{lemma:bisim-lang} and \ref{lemma:bisim-rpda-pda},
we obtain the following theorem.
\begin{theorem}
For a given $k$-NRPDA $\calA$, we can construct an NPDA $\calA'$
such that $\sigma(L(\calA))=L(\calA')$.
\end{theorem}
