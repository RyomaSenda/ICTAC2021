\section{Preliminaries}
Let $\Nat=\{1,2,\ldots\}$, $\Natz=\{0\} \cup \Nat$ and $[n]=\{1,\cdots,n\}$ for $n\in\Nat$.
For a set $A$, let $\scrP(A)$ be the power set of $A$,
let $A^*$ and $A^\omega$ be the sets of finite and infinite words over $A$, respectively.
We denote $A^+ = A^*\setminus\{\varepsilon\}$ and
$A^\infty = A^* \cup A^\omega$.
For a word $\alpha\in A^\infty$ over a set $A$,
let $\alpha(i)\in A$ be the $i$-th element of $\alpha$ ($i\geq 0$),
$\alpha(i:j)=\alpha(i)\alpha(i+1)\cdots\alpha(j-1)\alpha(j)$ for $i\geq j$
and $\alpha(i:)=\alpha(i)\cdots$ for $i\geq 0$.
Let $\<u, w\> = u(0)w(0)u(1)w(1)\cdots \in A^\infty$ for words $u,w\in A^\infty$ and $\<B, C\> = \{\<u, w\>\mid u\in B, w\in C\}$ for sets $B, C\subseteq A^\infty$.
By $|\beta|$, we mean the cardinality of $\beta$ if $\beta$ is a set
and the length of $\beta$ if $\beta$ is a finite sequence.
% We assume $\varepsilon$ represents the empty sequence, that is $|\varepsilon|=0$.

In this paper, disjoint sets $\Sigma_\dblI, \Sigma_\dblO$ and $\Gamma$ denote
a (finite) input alphabet, an output alphabet and a stack alphabet, respectively,
and $\Sigma = \Sigma_\dblI \cup \Sigma_\dblO$.
For a set $\Gamma$, let $\Com(\Gamma) = \{\pop,\skip\}\cup\{\push(z)\mid z\in \Gamma\}$ be the set of stack commands over $\Gamma$.

\subsection{Transition Systems}
\begin{definition}
A {transition system} (TS)
is $\calS=(S, s_0, A, E, \to_\calS, c)$ where
\begin{itemize}
\item $S$ is a (finite or infinite) set of states,
\item $s_0\in S$ is the initial state,
\item $A, E$ is (finite or infinite) alphabets such that $A\cap E = \emptyset$,
\item $\to_\calS\ \subseteq S\times(A\cup E)\times S$ is a set of transition relation, written as $s\to^a s'$ if $(s,a,s')\in\ \to_\calS$ and
\item $c: S \to [n]$ is a coloring function where $n\in \Nat$.
\end{itemize}
\end{definition}
An element of $A$ is an observable label and an element of $E$ is an internal label.
A \emph{run} of TS $\calS=(S, s_0, A, E, \to_\calS, c)$ is
a pair $(\rho, w)\in S^\omega \times (A\cup E)^\omega$ that satisfies
$\rho(0)=s_0$ and $\rho(i)\to^{w(i)}\rho(i+1)$ for $i\geq 0$.
Let $\mininf: S^\omega \to [n]$ be a minimal coloring function such that
$\mininf(\rho)=\min\{m\mid$ there exist an infinite number of $i\geq 0$ such that $c(\rho(i)) = m\}$.
We call $\calS$ deterministic if $s\to^a s_1$ and $s\to^a s_2$ implies $s_1=s_2$ for all $s,s_1,s_2\in S$ and $a\in A\cup E$.

For $w\in (A\cup E)^\omega$, let
$\ef(w) = a_0a_1\cdots\in A^\infty$ be the sequence obtained from $w$ by removing all symbols belonging to $E$.
Note that $\ef(w)$ is not always an infinite sequence even if $w$ is an infinite sequence.
We define the \emph{language} of $\calS$ as
$L(\calS)=\{\ef(w)\in A^\omega\mid$
there exists a run $(\rho,w)$ such that $\mininf(\rho)$ is even$\}$.
For $m\in\Natz$, we call an $\calS$ $m$-TS
if for every run $(\rho,w)$ of $\calS$,
$w$ contains no consecutive subsequence $w'\in E^{m+1}$.

% universal language as
% $L_U(\calS)=\{\ef(w)\in A^\omega\mid$
% all run $(\rho,w)$ satisfy $C(\rho)$ is even$\}$.
% We call $\calS$ nondeterministic if the language is defined as $L_N(\calS)$ and universal if the language is defined as $L_U(\calS)$.
% Note that $L_N(\calS)=L_U(\calS)$ holds when $\calS$ is deterministic.
