\appendix
\section{Appendix}

\subsection{A proof of Lemma \ref{lem: ef}}
\setcounter{lemma}{\ref{lem: ef}}
\addtocounter{lemma}{-1}
\begin{lemma}
For a given $m$-DPDA $\calA$,
we can construct a $0$-DPDA $\calA'$ such that
$L(\calA)=L(\calA')$
\end{lemma}
{\bf Proof.}\quad
For a given $m$-DPDA $\calA$,
we can construct an $2m$-DPDA $\calA'$ such that
$L(\calA)=L(\calA')$ and
$\calA'$ has no skip rule
by replacing every skip rule $(q,a,z)\to (q',\skip)$ of $\calA$
to a pair of push and pop rules
$(q,a,z)\to(q'',\push(z')), (q,\tau,z')\to(q',\pop)$ of $\calA'$
for $a\in \Sigma\cup\{\tau\}$.
Thus, we show the lemma for $m$-DPDA $\calA$ that has no skip rule
by the induction on $m$.
The case $m=0$ is obvious.
For an arbitrary $m$,
$m$-DPDA $\calA=(Q, Q_\dblI, Q_\dblO, q_0,z_0,\delta,c)$
over $\Sigma_\dblI, \Sigma_\dblO$ and $\Gamma$
can be converted to an
$(m-1)$-DPDA $\calA'$
over $\Sigma_\dblI, \Sigma_\dblO$ and $\Gamma^2$
such that $L(\calA)=L(\calA')$.
Let $\calA'=(Q\cup(Q\times\Gamma), Q_\dblI\cup(Q_\dblI\times\Gamma), Q_\dblO\cup(Q_\dblO\times\Gamma), (q_0,z_0), (z_0,z_0),\delta',c')$
such that
$c'(q)=c(q), c'((q,a))=c(q)$ for all $q\in Q, a\in\Sigma$ and
\begin{itemize}
\item
$(q,a,z_1)\to (q',\pop)\in\delta$
iff
$(q,a,(z_1,z_2))\to ((q',z_2),\pop)$,
$((q,z_1),a,(z_c,z'_c))\to (q',\skip)\in\delta'$
for all $z_c,z'_c\in\Gamma$.
\item
$(q,a,z_1)\to (q',\skip)\in\delta$
iff
$(q,a,(z_1,z_2))\to (q',\skip)$,
$((q,z_1),a,(z_c,z'_c))\to ((q',z_1),\skip)\in\delta'$
for all $z_c,z'_c\in\Gamma$.
\item
$(q,a,z_1)\to (q',\push(z'))\in\delta$
iff
$(q,a,(z_1,z_2))\to ((q',z'),\skip)$,
$((q,z_1),a,(z_c,z'_c))\to (q',\push((z',z_1)))\in\delta'$
for all $z_c,z'_c\in\Gamma$.
\end{itemize}
for $a\in\Sigma$, and
\begin{itemize}
\item
$(q,a,z_1)\to (q',\pop), (q',b,z_2)\to (q'',\pop)\in\delta$
iff
$(q,x,(z_1,z_2))\to (q'',\pop)$,
$((q,z_1),x,(z_2,z_c))\to ((q'',z_c),\pop)\in\delta'$
for all $z_c\in\Gamma$.
\item
$(q,a,z_1)\to (q',\push(z'))\in\delta', (q',b,z')\to (q'',\pop)\in\delta$
iff
$(q,x,(z_1,z_c))\to (q'',\skip)$,
$((q,z_1),x,(z_c,z'_c))\to ((q'',z_1),\skip)\in\delta'$
for all $z_c, z'_c\in\Gamma$.
\item
$(q,a,z_1)\to (q',\push(z')), (q',b,z')\to (q'',\push(z''))\in\delta$
iff
$(q,x,(z_1,z_c))\to (q'',\push((z'',z')))$,
$((q,z_1),x,(z_c,z'_c))\to ((q'',z''),\push(z',z_1))\in\delta'$
for all $z_c, z'_c\in\Gamma$.
\end{itemize}
where $a,b\in\Sigma\cup\{\tau\}$,
$x=a$ if $a\in\Sigma$ and $x=b$ otherwise.
As the definition of $\calA'$,
the ID $(q,z_1z_2z_3\cdots z_n)$ of $\calA$ corresponds to
an ID $(q, (z_1,z_2)\cdots(z_{n-1},z_n))$ of $\calA'$ if $n$ is even and
an ID $((q,z_1), (z_2,z_3)\cdots(z_{n-1},z_n))$ if $n$ is odd.
We can check $L(\calA)=L(\calA')$ by the induction on
the length of a sequence $w\in L(\calA)$.

\subsection{A full proof of Lemma \ref{lem: Lk}}
\setcounter{lemma}{\ref{lem: Lk}}
\addtocounter{lemma}{-1}
\begin{lemma}
$L_k = \{w\otimes\overa \mid w\in\Comp(\overa)\}$ is definable as the language of a $(2,k+2)$-DRPDA.
\end{lemma}
{\bf Proof.}\quad
Let $(2,k+2)$-DRPDA
$\calA_1=(Q_1, Q^\dblI_1, Q^\dblO_1, q^0_1, \delta_1, c_1)$
over $A_k^\dblI, A_k^\dblO$ and $D$ where
$Q_1=\{p, q\}\cup(\Asgn_k\times[k]\times\Com([k]))\cup[k]$, $Q^\dblI_1=\{p\}$, $Q^\dblO_1 = Q_1\setminus Q^\dblI_1$, $q^0_1 = p$,
$c_1(s)=2$ for every $s\in Q$ and $\delta_1$ consists of rules of the form
\begin{align}
(p, (a_\dblI, \tst), \tst\cup\tst')&\to(q, \{k+1\}, \skip)\tag{\ref{eq: Lk1}}\\
(q, (a_\dblO, \asgn, j, \com), \tst'')&\to((\asgn,j,\com), \{k+2\},\skip)\tag{\ref{eq: Lk2}}\\
((\asgn,j,\com), \tau, \{k+1\}\cup\tst'')&\to(j, \asgn, \com)\tag{\ref{eq: Lk3}}\\
(j, \tau, \{j, k+2\}\cup\tst'')&\to(p, \emptyset, \skip)\tag{\ref{eq: Lk4}}
\end{align}
for all $(a_\dblI, \tst)\in A_k^\dblI$, $(a_\dblO, \asgn, j, \com)\in A_k^\dblO$, $\tst'\subseteq \{k+1,k+2\}$ and $\tst''\in\Tst_{k+2}$.

We show $L(\calA_k) = L_k$.
For this proof, we redefine compatibility for finite sequences $w\in ((\Sigma_\dblI\times D)\cdot (\Sigma_\dblO\times D))^*$ and $\overa\in(A_k^\dblI\cdot A_k^\dblO)^*$.
We show the following claim.
\par\medskip\noindent
{\it Claim.} Let $n\in\Natz$ and $w\otimes\overa = ((a^\dblI_0,\skip), d^\dblI_0)((a^\dblO_0,\asgn_0,j_0, \com_0), d^\dblO_0)\cdots$ $\in ((A_k^\dblI\times D)\cdot(A_k^\dblO\times D))^*$
whose length is $2n$ and
$\rho = (\theta_0,u_0)(\theta_1, u_1)\cdots\in (\Theta_k\times D^*)^*$ whose length is $n+1$ and $(\theta_0, u_0)=(\theta^{k}_\bot, \bot)$.
Then,
$\rho$ is a witness of the compatibility between $w$ and $\overa$ iff
$(p,\theta'_0,u_0)\vdash^{w\otimes\overa(0:1)(\tau,d^\dblI_{0})(\tau,d^\dblO_{0})}(\theta'_1,u_1)\vdash^{w\otimes\overa(2:3)(\tau,d^\dblI_{1})(\tau,d^\dblO_{1})}\cdots\vdash^{w\otimes\overa(2n-2:2n-1)(\tau,d^\dblI_{n-1})(\tau,d^\dblO_{n-1})}(p,\theta'_{n},u_{n})$
where $\theta'_i\in \Theta_{k+2}$ ($i\in[n]$) satisfies $\theta'_i(j)=\theta_i(j)$ for $j\in[k]$.
\par\medskip\noindent
(Proof of the claim)
We show the claim by induction on $n$.
The case of $n=0$ is obvious.
We show the claim for arbitrary $n>0$ with the induction hypothesis.

We first show left to right.
By the induction hypothesis,
$(p,\theta'_0,u_0)\vdash^{w\otimes\overa(0:1)(\tau,d^\dblI_{0})(\tau,d^\dblO_{0})\cdots w\otimes\overa(2n-4:2n-3)(\tau,d^\dblI_{n-2})(\tau,d^\dblO_{n-2})}(p,\theta'_{n-1},u_{n-1})$ holds.
By the assumption, because $\rho$ is the witness,
(a) $\theta_{n-1}, d^\dblI_{n-1}, u_{n-1}(0) \models \tst_{n-1}$,
(b) $\theta_{n}=\theta_{n-1}[\asgn_{n-1}\leftarrow d^\dblI_{n-1}]$,
(c) $\theta_{n}(j_{n-1}) = d^\dblO_{n-1}$ and
(d) $u_{n}=\upds(u_{n-1},\theta_n,\com_{n-1})$ hold.
By the condition (a),
$\calA_k$ can do a transition
$(p,\theta'_{n-1},u_{n-1})\vdash^{w\otimes\overa(2n-2)}
(q,\theta^1_{n-1},u_{n-1})$ for unique $\theta^1_{n-1}\in\Theta_{k+2}$ by the rule $(p, (a^\dblI_{n-1}, \tst_{n-1}), \tst_{n-1}\cup\tst') \to (q, \{k+1\}, \skip)$ of the form (\ref{eq: Lk1}).
We can also say
$(q,\theta^1_{n-1},u_{n-1})\vdash^{w\otimes\overa(2n-1)}
((\asgn_{n-1}, j_{n-1}, \com_{n-1}),\theta^2_{n-1},u_{n-1})$ by the rule of the form (\ref{eq: Lk2}). Note that $\theta^2_{n-1}(j)=\theta_{n-1}(j)$ if $j\in[k]$, $\theta^2_{n-1}(k+1)=d^\dblI_{n-1}$ and
$\theta^2_{n-1}(k+2)=d^\dblO_{n-1}$.
$((\asgn_{n-1}, j_{n-1}, \com_{n-1}),\theta^2_{n-1},u_{n-1})\vdash^{(\tau,d^\dblI_{n-1})}(j_{n-1},\theta^3_{n-1},u_{n})$ is also valid transition of $\calA_k$ of the form (\ref{eq: Lk3}) by the conditions (b) and (d)
where $\theta_3^{n-1}(j)=\theta_{n}(j)$
for $j\in[k]$ and $\theta^3_{n-1}(k+2)=d^\dblO_{n-1}$.
By the condition (c), $\theta_3^{n-1}(j_{n-1})=\theta_3^{n-1}(k+1)=d^\dblO_{n-1}$ holds.
Thus, a transition $(j_{n-1},\theta^3_{n-1},u_{n})\vdash^{(\tau,d^\dblO_{n-1})}(p,\theta'_{n},u_{n})$ is valid with the rule of the form (\ref{eq: Lk4}).
In conclusion,
$(p,\theta'_{n-1},u_{n-1})\vdash^ {w\otimes\overa(2n-2:2n-1)(\tau,d^\dblI_{n-1})(\tau,d^\dblO_{n-1})}(p,\theta'_{n},u_{n})$
holds, and with the induction hypothesis, we obtain the left to right of the claim.

Next, we prove right to left. By the assumption,
$(p,\theta'_{n-1},u_{n-1})\vdash^ {w\otimes\overa(2n-2:2n-1)(\tau,d^\dblI_{n-1})(\tau,d^\dblO_{n-1})}(p,\theta'_{n},u_{n})$
holds. By checking four transition rules that realize the above transition relation, we can obtain that $\rho(n-1), \rho(n), w\otimes\overa(2n-2)$ and $w\otimes\overa(2n-1)$ satisfies the conditions (a) to (d) described in the previous paragraph. Thus, by the induction hypothesis,
we obtain $\rho$ is a witness of the compatibility between $w$ and $\overa$.
\begin{flushright}
(end of the proof of the claim)
\end{flushright}
By the claim,
$w\otimes\overa \in L_k\Leftrightarrow$
there exists a witness $(\theta_0,u_0)(\theta_1, u_1)\cdots\in (\Theta_k\times D^*)^\omega$ of $w$ and $\overa\Leftrightarrow$
there exists a run
$(p,\theta'_0,u_0)\vdash^{w\otimes\overa(0:1)(\tau,d^\dblI_{0})(\tau,d^\dblO_{0})}(\theta'_1,u_1)\vdash^{w\otimes\overa(2:3)(\tau,d^\dblI_{1})(\tau,d^\dblO_{1})}\cdots$ of $\calA\Leftrightarrow$
$w\otimes\overa \in L(\calA_k)$ holds
for all $w\otimes\overa \in \DW(A_k^\dblI, A_k^\dblO, D)$.




\subsection{A full proof of Lemma \ref{lem: LkS}}
\setcounter{lemma}{\ref{lem: LkS}}
\addtocounter{lemma}{-1}
\begin{lemma}
For a specification $\calS$ defined by some visibly $k'$-DRPDA,
$L_{\overS, k} = \{w\otimes\overa \mid w\in\Comp(\overa)\cap \overS\}$
is definable as the language of a $(4,k+k'+4)$-DRPDA.
\end{lemma}
{\bf Proof.}\quad
Let $L_{\overS} = \{w\otimes\overa\mid w\in\overS, \overa\in(A_k^\dblI\cdot A_k^\dblO)^\omega\}$.
We can construct a visibly $k'$-DRPDA
$\calA_2=(Q_2, Q^\dblI_2, Q^\dblO_2, q^0_2, \delta_2, c_2)$
over $A^\dblI_{k}, A^\dblO_{k}$ and $D$
such that $L(\calA_2)= L_{\overS}$.
Let $\calA_1$ be the $(2,k+2)$-DRPDA
such that $L(\calA_1)=L_k$, which is given in Lemma \ref{lem: Lk}.
Because $L_{\overS, k} = L_{\overS} \cap L_k$, it is enough to show that we can construct a $(4,k+k'+4)$-DRPDA $\calA$
over $A^\dblI_{k}, A^\dblO_{k}$ and $D$
such that $L(\calA)=L(\calA_1)\cap L(\calA_2)$.

To construct $\calA$, we use the facts that
$c_1(q)$ is even for every $q\in Q_1$ and
$\delta_1$ consists of several groups of three consecutive rules having the following forms:
\begin{align}
(q_1, a, \tst_1) &\to (q_2,\asgn_1, \skip) \tag{\ref{eq: Lk2}'}\\
(q_2, \tau, \tst_2) &\to (q_3,\asgn_2, \com_1) \tag{\ref{eq: Lk3}'}\\
(q_3, \tau, \tst_3) &\to (q_4,\asgn_3, \skip) \tag{\ref{eq: Lk4}'}.
\end{align}
Note that $\vis(a)=v(\com_1)$ always holds for every consecutive rules.
$(\ref{eq: Lk2}'), (\ref{eq: Lk3}')$ and $(\ref{eq: Lk4}')$ correspond to $(\ref{eq: Lk2}), (\ref{eq: Lk3})$ and $(\ref{eq: Lk4})$, respectively, and $(\ref{eq: Lk1})$
is also converted to three consecutive rules like $(\ref{eq: Lk2}')$-$(\ref{eq: Lk4}')$
by adding dummy $\tau$ rules.

We let $k_1=k+2$ and $k_2=k'$.
We construct $(4,k_1+k_2+2)$-DRPDA
$\calA=(Q_\dblI\cup Q_\dblO\cup\{q_0\}, Q_\dblI\cup\{q_0\}, Q_\dblO, q_0, \delta, c)$
where $Q_\dblI=Q^\dblI_1\times Q^\dblI_2\times[5]$,
$Q_\dblO=Q^\dblO_1\times Q^\dblO_2\times[5]$.
$c$ is defined as $c(q_0)=1$ and
$c((q_1,q_2,i))=c_2(q_2)$ for all $(q_1,q_2,i)\in Q$.
$\delta$ has first $\tau$ rule $(q_0,\tau,[k]\cup\{\top\})\to((q^1_0, q^2_0,1), \push(1))$.
For all rules
\begin{align}
(q_1, a, \tst_1) &\to  (q_2,\asgn_1, \skip)\in\delta_1\tag{\ref{eq: LkS1}}\\
(q_2, \tau, \tst_2) &\to  (q_3,\asgn_2, \com_1)\in\delta_1\tag{\ref{eq: LkS2}}\\
(q_3, \tau, \tst_3) &\to  (q_4,\asgn_3, \skip)\in\delta_1\tag{\ref{eq: LkS3}}\\
(q, a, \tst) &\to (q',\asgn, \com)\in\delta_2\tag{\ref{eq: LkS4}}
\end{align}
(note that $v(\com_1)=\vis(a)=v(\com)$ always hold) for $a\in A^\dblI_{k}\cup A^\dblO_{k}$,
let $\tst^{+ k_1}=\{i+k_1\mid i\in\tst\}\cup\{\top\mid\top\in\tst\setminus[k_1]\}, \asgn^{+ k_1}=\{i+k_1\mid i\in\asgn\}$ and $\com^{+ k_1} = \push(j+k_1)$ if $\com=\push(j)$ and $\com^{+ k_1} = \com$ otherwise, then $\delta$ consists of the rules
\begin{align}
((q_1, q, 1), \tau, \T^{k_1+k_2+2}_{1}\cup\{\top\}) &\to ((q_1, q, 2), \{k_1+k_2+1\}, \pop)\tag{\ref{eq: LkS5}}\\
((q_1, q, 2), \tau, \T^{k_1+k_2+2}_{1}\cup\{\top\})&\to ((q_1, q, 3), \{k_1+k_2+2\}, \push(k_1+k_2+1))\tag{\ref{eq: LkS6}}\\
((q_1, q, 3), a,
\tst_1\cup ((\tst^{+ k_1}\ \setminus\ &\{\top\})\cup\Top))\nonumber\\
&\to ((q_2, q', 4), \asgn_1\cup\asgn^{+ k_1}, \com^{+ k_1})\tag{\ref{eq: LkS7}}\\
((q_2, q', 4), \tau,((\tst_2\setminus\{\top\})\ \cup\ &\Top')\cup\T^{k_1+k_2+1}_{k_1+1})\nonumber\\
&\to ((q_3, q', 5), \asgn_2, \com_1)\tag{\ref{eq: LkS8}}\\
((q_3, q', 5), \tau, \tst_3\cup\T^{k_1+k_2+2}_{k_1+1}) &\to ((q_4, q', 1), \asgn_3, \skip)\tag{\ref{eq: LkS9}}
\end{align}
for all $\T^{j}_{i}\in\Tst_{j}\setminus\Tst_{i-1}$ (we assume $\Tst_0=\emptyset$) where
$\Top=\{k_1+k_2+2\}$ ($\Top'=\{k_1+k_2+1\}$) if $\top\in\tst$ ($\top\in\tst_2$) and $\Top=\emptyset$ ($\Top'=\emptyset$) otherwise.

Fig. \ref{fig: lem_LkS} indicates an example of transitions
from $(q_1,q,1)$ to $(q_4,q',1)$ with updating
contents of registers and stack.
The first to $k_1$-th registers simulate
the registers of $\calA_1$,
$(k_1+1)$-th to $(k_1+k_2)$-th register simulate
the registers of $\calA_2$ and
$(k_1+k_2+1)$-th and $(k_1+k_2+2)$-th registers
is for keeping the first and second stack top contents, respectively.
The stack contents of $\calA$ simulates that of $\calA_1$ and $\calA_2$ by restoring contents of stacks of $\calA_1$ and $\calA_2$ alternately.

The transition rules (\ref{eq: LkS5}) and (\ref{eq: LkS6})
is for restoring two stack data values from the top
to $(k_1+k_2+1)$-th and $(k_1+k_2+2)$-th registers.
The transition rule (\ref{eq: LkS7})
is for updating states, registers and stacks
by simulating the rules (\ref{eq: LkS1}) and (\ref{eq: LkS4}).
Because the first to $k_2$-th registers of $\calA_2$ is simulated by $(k_1+1)$-th to $(k_1+k_2)$-th registers of $\calA$,
we use $\tst^{+k_1}$, $\asgn^{+k_1}$ and $\com^{+k_1}$
instead of $\tst$, $\asgn$ and $\com$, and
replace $\top$ in $\tst^{+k_1}$ to $k_1+k_2+2$.
The transition rules (\ref{eq: LkS8}) and (\ref{eq: LkS9})
simulates the rules (\ref{eq: LkS2}) and (\ref{eq: LkS3}), respectively.

For two assignments $\theta_1\in\Theta_{k_1}$ and $\theta_2\in\Theta_{k_2}$,
let $[\theta_1,\theta_2,d,d']\in\Theta_{k_1+k_2+2}$
be the assignment such that
$[\theta_1,\theta_2,d,d'](i)=\theta_1(i)$ if $i\in[k_1]$,
$[\theta_1,\theta_2,d,d'](i)=\theta_2(i)$ if $k_1+1\leq i\leq k_2$,
$[\theta_1,\theta_2,d,d'](k_1+k_2+1)=d$ and
$[\theta_1,\theta_2,d,d'](k_1+k_2+2)=d'$.
To prove $L(\calA)=L(\calA_1)\cap L(\calA_2)$,
we show the following claim.
\par\medskip\noindent
{\it Claim.}
For all $n\in\Natz$ and $w\in ((A_{k}^\dblI \cup A_{k}^\dblO) \times D)^n$,
there exists sequences of transitions
$(q_0^1,\theta^1_0,u^1_0)\vdash_{\calA_1}^{w(0)(\tau,d_0)(\tau,d'_0)}
(q_1^1,\theta^1_1,u^1_1)\vdash_{\calA_1}^{w(1)(\tau,d_1)(\tau,d'_1)} \cdots \vdash_{\calA_1}^{w(n-1)(\tau,d_{n-1})(\tau,d'_{n-1})}
(q^1_n,\theta^1_n,u^1_n)$ and
$(q_0^2,\theta^2_0,u^2_0)\vdash_{\calA_2}^{w(0)}
(q_1^2,\theta^2_1,u^2_1)\vdash_{\calA_2}^{w(1)} \cdots$\\
$\vdash_{\calA_2}^{w(n-1)} (q^2_n,\theta^2_n,u^2_n)$
iff
$(q_0, \theta^{k_1+k_2+2}_\bot, \bot)
\vdash_\calA^{(\tau,\bot)}
((q^1_0,q^2_0,1),[\theta^1_0,\theta^2_0,d^1_0,d^2_0],\<u^1_0,u^2_0\>)$
\\
$\vdash_\calA^{(\tau,u^1_0(0))(\tau,u^2_0(0))w(0)(\tau,d_0)(\tau,d'_0)}
((q^1_1,q^2_1,1),[\theta^1_1,\theta^2_1,d^1_1,d^2_1],\<u^1_1,u^2_1\>)
\vdash_\calA^{(\tau,u^1_1(0))(\tau,u^2_1(0))w(1)(\tau,d_1)(\tau,d'_1)}
\cdots
\vdash_\calA^{(\tau,u^1_{n-1}(0))(\tau,u^2_{n-1}(0))w(n-1)(\tau,d_{n-1})(\tau,d'_{n-1})}
((q^1_n,q^2_n,1),[\theta^1_n,\theta^2_n,d^1_n,d^2_n],\<u^1_n,u^2_n\>)$
holds where $b\in\{1,2\}, i\in[n],$
$\theta^b_0=\theta^{k_b}_{\bot}, u^b_0=\bot,$
$q^b_i\in Q_b$, $\theta^b_i\in\Theta_{k_b}$, $u^b_i\in D^*$ and
$d_{i-1}, d'_{i-1}\in D$.
\par\medskip\noindent
(Proof of the claim)
We show the claim by the induction on $n$.
The case $n=0$ is obvious.

We first show left to right.
By induction hypothesis,
$(q_0, \theta^\calA_\bot, \bot)
\vdash_\calA^{(\tau,\bot)}
((q^1_0,q^2_0,1),[\theta^1_0,\theta^2_0,d^1_0,d^2_0],\<u^1_0,u^2_0\>)
\vdash_\calA^{(\tau,u^1_0(0))(\tau,u^2_0(0))w(0)(\tau,d_0)(\tau,d'_0)}
\cdots$
\\
$\vdash_\calA^{(\tau,u^1_{n-2}(0))(\tau,u^2_{n-2}(0))w(n-2)(\tau,d_{n-2})(\tau,d'_{n-2})}
((q^1_{n-1},q^2_{n-1},1),[\theta^1_{n-1},\theta^2_{n-1},d^1_{n-1},d^2_{n-1}],$
\\
$\<u^1_{n-1},u^2_{n-1}\>)$
holds.
Also, by the assumption,
$(q^1_{n-1},\theta^1_{n-1},u^1_{n-1})
\vdash_{\calA_1}^{w(n-1)}
(q',\theta',u^1_{n-1})
\vdash_{\calA_1}^{(\tau,d)}
(q'',\theta'',u^1_n)
\vdash_{\calA_1}^{(\tau,d')}
(q^1_n,\theta^1_n,u^1_n)$
and
$(q^2_{n-1},\theta^2_{n-1},u^2_{n-1})
\vdash_{\calA_2}^{w(n-1)}
(q^2_n,\theta^2_n,u^2_n)$
for some $q',q''\in Q_1, \theta',\theta''\in\Theta_{k_1}$ and
$d,d'\in D$,
and let the following be the rules used in these transitions.
\begin{align}
(q_{n-1}, (a,v(\com_1)), \tst_1) &\to(q',\asgn_1, \skip)\in\delta_1 \tag{\ref{eq: LkS1}'}\\
(q', \tau, \tst_2) &\to(q'',\asgn_2, \com_1)\in\delta_1 \tag{\ref{eq: LkS2}'}\\
(q'', \tau, \tst_3) &\to(q_{n},\asgn_3, \skip)\in\delta_1 \tag{\ref{eq: LkS3}'}\\
(q_{n-1}, (a,v(\com)), \tst) &\to(q_{n},\asgn, \com)\in\delta_2 \tag{\ref{eq: LkS4}'}
\end{align}
We can check the follows.
\begin{align*}
&((q^1_{n-1},q^2_{n-1},1),[\theta^1_{n-1},\theta^2_{n-1},d^1_{n-1},d^2_{n-1}],\<u^1_{n-1},u^2_{n-1}\>)
\vdash_\calA^{(\tau,u^1_{n-1}(0))(\tau,u^2_{n-1}(0))}\\
&((q^1_{n-1},q^2_{n-1},3),[\theta^1_{n-1},\theta^2_{n-1},u^1_{n-1}(0),u^1_{n-1}(0)],\<u^1_{n-1},u^2_{n-1}\>)
\vdash_\calA^{w(n-1)}\\
&((q^1_{n-1},q^2_{n-1},4),[\theta^1_{n},\theta',u^1_{n-1}(0),u^1_{n-1}(0)],\<u^1_{n-1},u^2_{n}\>)
\vdash_\calA^{(\tau,d_{n-1})}\\
&((q^1_{n-1},q^2_{n-1},5),[\theta^1_{n},\theta'',u^1_{n-1}(0),u^1_{n-1}(0)],\<u^1_{n},u^2_{n}\>)
\vdash_\calA^{(\tau,d'_{n-1})}\\
&((q^1_{n},q^2_{n},1),[\theta^1_{n},\theta^2_{n},u^1_{n-1}(0),u^1_{n-1}(0)],\<u^1_{n},u^2_{n}\>)
\end{align*}
Thus, the right side of the claim holds.
In a similar way, we can also show right to left.
\begin{flushright}
(end of the proof of the claim)
\end{flushright}
By the claim,
$w\in L(\calA_1)\cap L(\calA_2)$
$\Leftrightarrow$
there exists runs
$(q_0^1,\theta^1_0,u^1_0)\vdash_{\calA_1}^{w(0)(\tau,d_0)(\tau,d'_0)}
(q_1^1,\theta^1_1,u^1_1)\vdash_{\calA_1}^{w(1)(\tau,d_1)(\tau,d'_1)} \cdots$ and
$(q_0^2,\theta^2_0,u^2_0)\vdash_{\calA_2}^{w(0)}
(q_1^2,\theta^2_1,u^2_1)\vdash_{\calA_2}^{w(1)} \cdots$
that satisfies the minimum number appearing in
the sequence $q^1_0, q^1_1, \cdots$ infinitely is even.
$\Leftrightarrow$
there exists a run
$(q_0, \theta^\calA_\bot, \bot)
\vdash_\calA^{(\tau,\bot)}
((q^1_0,q^2_0,1),[\theta^1_0,\theta^2_0,d^1_0,d^2_0],\<u^1_0,u^2_0\>)
\vdash_\calA^{(\tau,u^1_0(0))(\tau,u^2_0(0))w(0)(\tau,d_0)(\tau,d'_0)}
((q^1_1,q^2_1,1),$
\\
$[\theta^1_1,\theta^2_1,d^1_1,d^2_1],\<u^1_1,u^2_1\>)
\vdash_\calA^{(\tau,u^1_1(0))(\tau,u^2_1(0))w(1)(\tau,d_1)(\tau,d'_1)}
\cdots$
that satisfies the minimum number appearing in
the sequence $(q^1_0,q^2_0,1), (q^1_0,q^2_0,2) \cdots, (q^1_1,q^2_1,1), \cdots$ infinitely is even.
$\Leftrightarrow$
$w\in L(\calA)$ holds.
