\section{Preliminaries}
Let $\Nat=\{1,2,\ldots\}$, $\Natz=\{0\} \cup \Nat$ and $[n]=\{1,\cdots,n\}$ for $n\in\Nat$.
For a set $A$, let $\scrP(A)$ be the power set of $A$,
let $A^*$ and $A^\omega$ be the sets of finite and infinite words over $A$, respectively.
We denote $A^+ = A^*\setminus\{\varepsilon\}$ and
$A^\infty = A^* \cup A^\omega$.
For a word $\alpha\in A^\infty$ over a set $A$,
let $\alpha(i)\in A$ be the $i$-th element of $\alpha$ ($i\geq 0$),
$\alpha(i:j)=\alpha(i)\alpha(i+1)\cdots\alpha(j-1)\alpha(j)$ for $i\geq j$
and $\alpha(i:)=\alpha(i)\cdots$ for $i\geq 0$.
Let $\<u, w\> = u(0)w(0)u(1)w(1)\cdots \in A^\infty$ for words $u,w\in A^\infty$ and $\<B, C\> = \{\<u, w\>\mid u\in B, w\in C\}$ for sets $B, C\subseteq A^\infty$.
By $|\beta|$, we mean the cardinality of $\beta$ if $\beta$ is a set
and the length of $\beta$ if $\beta$ is a finite sequence.
% We assume $\varepsilon$ represents the empty sequence, that is $|\varepsilon|=0$.
For a function $f:A\to B$ from a set $A$ to a set $B$,
let $f(w)=f(w(0))f(w(1))\ldots$ for a word $w\in A^{\infty}$
and let $f(L)=\{f(w)\mid w\in L\}$ for a set $L\subseteq A^{\infty}$
of words.
Let $\fst$ and $\scnd$ be the functions such that $\fst((a,b))=a$ and $\scnd((a,b))=b$ for any pair $(a,b)$.
Let $\mathit{id}$ be the identity function; i.e.,
$\mathit{id}(a)=a$ for any $a$.

\subsection{Transition Systems}
\begin{definition}
A {transition system} (TS)
is $\calS=(S, s_0, A, E, \to_\calS, c)$ where
\begin{itemize}
\item $S$ is a (finite or infinite) set of states,
\item $s_0\in S$ is the initial state,
\item $A, E$ are (finite or infinite) alphabets such that $A\cap E = \emptyset$,
\item $\to_\calS\ \subseteq S\times(A\cup E)\times S$ is a transition relation, written as $s\to^a s'$ if $(s,a,s')\in\ \to_\calS$ and
\item $c: S \to [n]$ is a coloring function where $n\in \Nat$.
\end{itemize}
\end{definition}
An element of $A$ is an observable label and an element of $E$ is an internal label.
A \emph{run} of TS $\calS=(S, s_0, A, E, \to_\calS, c)$ is
a pair $(\rho, w)\in S^\omega \times (A\cup E)^\omega$ that satisfies
$\rho(0)=s_0$ and $\rho(i)\to^{w(i)}\rho(i+1)$ for $i\geq 0$.
Let $\mininf: S^\omega \to [n]$ be the minimal coloring function such that
$\mininf(\rho)=\min\{m\mid$ there exist an infinite number of $i\geq 0$ such that $c(\rho(i)) = m\}$.
We call $\calS$ deterministic if $s\to^a s_1$ and $s\to^a s_2$ implies $s_1=s_2$ for all $s,s_1,s_2\in S$ and $a\in A\cup E$.

For $w\in (A\cup E)^\omega$, let
$\ef(w) = a_0a_1\cdots\in A^\infty$ be the sequence obtained from $w$ by removing all symbols belonging to $E$.
Note that $\ef(w)$ is not always an infinite sequence even if $w$ is an infinite sequence.
We define the \emph{language} of $\calS$ as
$L(\calS)=\{\ef(w)\in A^\omega\mid$
there exists a run $(\rho,w)$ such that $\mininf(\rho)$ is even$\}$.
For $m\in\Natz$, we call $\calS$ an $m$-TS
if for every run $(\rho,w)$ of $\calS$,
$w$ contains no consecutive subsequence $w'\in E^{m+1}$.



Consider two TSs
$\calS_1=(S_1,s_{01},A_1,E_1,{\to_{\calS_1}},c_1)$ and
$\calS_2=(S_2,s_{02},A_2,E_2,{\to_{\calS_2}},\allowbreak c_2)$
and
a function $\sigma:(A_1\cup E_1)\to(A_2\cup E_2)$.
We call $R\subseteq S_1\times S_2$
a $\sigma$-\emph{bisimulation relation} from $\calS_1$ to $\calS_2$
if $R$ satisfies the followings:
\begin{enumerate}[label=(\arabic*)]
\item $(s_{01},s_{02})\in R$.
\item For any $s_1,s'_1\in S_1$, $s_2\in S_2$, and
  $a_1\in A_1\cup E_1$,
  if $s_1\to_{\calS_1}^{a_1} s'_1$ and $(s_1,s_2)\in R$,
  then $\exists s'_2\in S_2: s_2\to_{\calS_2}^{\sigma(a_1)} s'_2$
  and $(s'_1,s'_2)\in R$.
\item For any $s_1\in S_1$, $s_2,s'_2\in S_2$, and
  $a_2\in A_2\cup E_2$,
  if $s_2\to_{\calS_2}^{a_2} s'_2$ and $(s_1,s_2)\in R$,
  then $\exists s'_1\in S_1$, $\exists a_1\in A_1\cup E_1:
  \sigma(a_1)=a_2$ and
  $s_1\to_{\calS_1}^{a_1} s'_1$
  and $(s'_1,s'_2)\in R$.
\item If $(s_1,s_2)\in R$, then $c_1(s_1)=c_2(s_2)$.
\end{enumerate}
%
We say $\calS_1$ is $\sigma$-\emph{bisimilar} to
$\calS_2$ if there exists a $\sigma$-bisimulation relation
from $\calS_1$ to $\calS_2$.
%
We call $R$ a bisimulation relation
if $R$ is an $\mathit{id}$-bisimulation relation.
We say $\calS_1$ and $\calS_2$ are bisimilar if
$\calS_1$ is $\mathit{id}$-bisimilar to $\calS_2$.

The following lemma can be proved by definition.
\begin{lemma}\label{lemma:bisim-lang}
If
$\calS_1=(S_1,s_{01},A_1,E_1,{\to_{\calS_1}},c_1)$ is
$\sigma$-bisimilar to
$\calS_2=(S_2,s_{02},A_2,\allowbreak E_2,{\to_{\calS_2}},c_2)$
for
a function $\sigma:(A_1\cup E_1)\to(A_2\cup E_2)$
that satisfies $a\in A_1 \Leftrightarrow \sigma(a)\in A_2$
for any $a\in A_1\cup E_1$,
then
$\sigma(L(\calS_1))=L(\calS_2)$.
\end{lemma}






% universal language as
% $L_U(\calS)=\{\ef(w)\in A^\omega\mid$
% all run $(\rho,w)$ satisfy $C(\rho)$ is even$\}$.
% We call $\calS$ nondeterministic if the language is defined as $L_N(\calS)$ and universal if the language is defined as $L_U(\calS)$.
% Note that $L_N(\calS)=L_U(\calS)$ holds when $\calS$ is deterministic.
