\section{Introduction}
\noindent
Reactive synthesis is a method of synthesizing a system that satisfies a given specification
representing the input-output relation of the system.
A specification is formally a subset of an infinite alternate sequences of input and output symbols.
When an environment or a user gives inputs $i_0, i_1, \ldots$ in this order
to a system, the latter is required to emit a sequence
of outputs $o_0, o_1, \ldots$ such that $(i_0, o_0)(i_1, o_1)\cdots \in S$.
The realizability problem is to decide whether for a given specification $S$,
there exists a reactive system satisfying $S$, and if exists, to generate such a system
(called an implementation of $S$).

Studies on reactive synthesis has its origin in 1960s and has been one of central topics
in formal methods \cite{CGP01}.
Among them, B\"{u}chi and Landweber \cite{BL69} showed EXPTIME-completeness of the problem
when a specification is given by a finite $\omega$-automaton (a finite automaton on infinite words).
Pnueli and Rosner \cite{PR89} showed 2EXPTIME-completeness of the problem
when a specification is given by an LTL formula.
We can find an excellent tutorial and survey of the previous studies
on the synthesis problem in \cite{BCJ18}.
The standard approach to the problem is as follows.
%Assume that a specification is given by a deterministic finite automaton ${\cal A}$
%on infinite words of input and output symbols.
Assume that, for example, a specification is given as a deterministic $\omega$-automaton ${\cal A}$.
We convert ${\cal A}$ to a tree automaton (or equivalently, a parity game) ${\cal B}$
by separating the input and output streams.
Then, we test whether $L({\cal B})\neq\emptyset$ (or equivalently, there is a winning strategy
for player I in ${\cal B}$).
The answer to the problem is affirmative if and only if $L({\cal B})\not=\emptyset$,
and any $t\in L({\cal B})$ (or any winning strategy for ${\cal B}$)
is an implementation of the specification.

Classical models such as a finite automaton cannot deal with
objects from an infinite set (called data values).
However, when we add an ability of manipulating data values to such a classical model,
the model easily becomes Turing-complete and basic problems become undecidable.
Register automaton (RA) is an extension of finite automaton by adding limited ability of
manipulating data values \cite{KF94,NSV04,Se06}.
An RA has a finite number of registers for storing data values taken from an input word and
it can compare the contents of its registers with the current input data value
to determine the next transition.
RA inherits some good properties from finite automaton such as the closure under some
language opeations.
The membership and emptiness are decidable for RA while the universality is undecidable.
With the increase of interest in RA, the realizability problem for register models
has been investigated recently \cite{ESK14,KMB18,EFR19,KK19}.
A specification is given by an RA and
an implementation is represented by a register transducer (abbreviated as RT).

Pushdown automaton (PDA) is a simple model for recursive programs.
Model checking algorithms for pushdown systems (PDA without input)
has been extensively studied \cite{Gr67,Wa96,BEM97,EHRS00,EKS01}.
Extensions by adding registers are also done for PDA \cite{CL15,MRT17,STS21},
called register pushdown automaton (RPDA).
Since many real-world programs are recursive and also manipulate data values,
it is important to investigate the realizability problem for PDA/RPDA.
\medskip\par
This paper first extends the realizability problem to PDA and pushdown transducer.
%%% The main difficulty to solve the problem for PDA is that
A pushdown transducer (PDT) is a deterministic PDA with output.
PDT serves as a model of a recursive program that emits outputs
according to the inputs given from its environment.
The main difficulties to solve the realizability problem in this setting
come from the facts that
the class of languages recognized by nondeterministic PDA (NPDA) (i.e., context-free languages)
does not have the closure properties under some set operations and
some relevant decision problems such as the universality are undecidable for PDA.
To avoid these difficulties, this paper mainly considers deterministic PDA (DPDA).
(Note that, PDT is deterministic by definition.)
In section 4, we show that the realizability problem for DPDA is decidable
while the problem is undecidable for NPDA.
The former is proved by using the well-known property that
the (two-players zero-sum parity) pushdown game is decidable and
a winning strategy of the game can be constructed as a PDT \cite{Wa96}.

Then, the paper moves to the realizability problem for RPDA and register PDT.
In section 5, we introduce RPDA and register PDT (RPDT).
Both RPDA and RPDT read a data word, which is an infinite sequence of
a pair $(a,d)$ of a symbol $a$ from a finite alphabet and a data value $d$ from
an infinite set.
We show that the projection of the language recognized by a nondeterministic RPDA
onto the finite alphabet can be recognized by a nondeterministic PDA,
when we assume the freshness of input data values \cite{Tz10} for RPDA.

In section 6, we discuss the realizability problem for deterministic RPDA (DRPDA) and RPDT.
Our approach is to reduce the problem for DRPDA to the problem for DPDA.
In this reduction, we want to convert a given DRPDA to a DPDA that recognizes
the projection onto the finite alphabet of the former language by using the method
in section 5.
For this purpose, we further assume that a given DPDA has the visibility
on the guard condition inherited from the DRPDA as well as the visibility
of stack operation (see \cite{AM04} for the visibility of stack operation).
Finally, we show that the realizability problem is decidable for visibly DRPDA and RPDT.
\medskip\par\noindent
{\bf Related work}~
RA is frequently used as
computation models for querying structured data such as XML documents
and graph databases \cite{LV12,LTV15}.
LTL with the freeze quantifier (LTL$\downarrow$) \cite{DL09,DLN07} is
an extension of LTL by adding the ability of memorizing and comparing data values.
LTL$\downarrow$ has the strong relationship with two-way alternate extension of RA.
Two-variable first-order logic on data words \cite{BDMSS11}(also see \cite{Bo02})
is another famous logic for data values,
whose expressive power is incomparable with LTL$\downarrow$.
Other extensions of RA are found in \cite{BKL14,CLTW17,DFSS19,BCGK12}.
Nominal automaton \cite{BKL14} is an extension of finite automaton
by using nominal sets, which are infinite sets having finite orbits
of group actions and equivariant functions among them.

When considering the realizability for RA,
the difficulty is to identify the number of registers
needed for implementing the specification, i.e., the number of registers of RT.
If the upperbound of the registers of RT is not given as an input,
the realizability problem is undecidable for both of nondeterministic and
universal RA \cite{EFR19}.
If the upperbound of the registers is {\em a priori} known,
the problem (called the bounded realizability
problem) is shown to be decidable in EXPTIME for universal RA \cite{KK19}.
In \cite{EFR19}, it is shown that the bounded realizability problem remains undecidable
for nondeterministic RA (NRA)
and becomes decidable in 2EXPTIME for a subclass called test-free NRA.

Extensions by adding registers are also done for PDA \cite{CL15,MRT17,STS21} and
context-free languages \cite{CK98,STS18,STS19},
called register pushdown automaton (RPDA) and register context-free grammar (RCFG), respectively.
The expressive powers of RPDA and RCFG are the same.
RPDA is a natural model for recursive programs with data values while
RCFG has an advantage such that explicit representation of pushdown stack is not needed.
For other extensions of PDA and verification of them, see \cite{ST12,XO13,RBB10}.
